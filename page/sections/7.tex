\section{第七题:安全管理}

\subsection{题目}

\begin{enumerate}
  \item 请建立SQL Server登录名sql\_user1,并映射至同名用户名。其中登录密码为'nulibeikao'。
  \item 对象权限,授予sql\_user1用户在student表中的插入,更新,查询权利。
  \item 语句权限,允许sql\_user1用户在数据库上创建视图、存储过程的权限。
  \item 语句权限,拒绝sql\_user1用户在数据库上创建表的权限
  \item 对象权限,拒绝sql\_user1用户在sc表上删除数据的权利
\end{enumerate}

\subsection{解析}

\qquad 这一章的内容基本都可以通过SSMS软件进行图形化操作,详见书本第九章P202。
这里主要介绍如何通过指令操作。

\subsubsection{问题一}

\textbf{题目} \emph{请建立SQL Server登录名sql\_user1,并映射至同名用户名。其中登录密码为'nulibeikao'。}

\qquad 分析创建和映射数据库用户的简单模板。

\begin{mdframed}[backgroundcolor=gray!10]
\begin{verbatim}
-- 创建数据库用户
create user 用户名
[with Password=密码]
[DEFAULT_DATABASE=默认数据库]

-- 创建数据库登录名
create login 登录名
[with Password=密码]

-- 映射登录名到数据库用户
create user 用户名
for login 登录名
\end{verbatim}
\end{mdframed}

\qquad \textbf{用户名、登录名、映射关系简介:}

\begin{itemize}
  \item \textbf{登录名(Login)}:是SQL Server服务器级别的安全主体,用于连接到SQL Server实例。登录名存储在master数据库中,是进入SQL Server的"钥匙"。

  \item \textbf{用户名(User)}:是数据库级别的安全主体,存在于特定的数据库中。用户名决定了在该数据库内能执行哪些操作。

  \item \textbf{映射关系}:登录名和用户名之间需要建立映射关系,这样登录名才能访问特定的数据库。一个登录名可以映射到多个数据库中的不同用户名。一个登录名在一个数据库中只能有有唯一一个数据库用户与之对应。

  \item \textbf{权限层次}:
    \begin{itemize}
      \item 服务器级权限 $\rightarrow$ 登录名
      \item 数据库级权限 $\rightarrow$ 用户名
    \end{itemize}
\end{itemize}

\qquad 简单来说:\textbf{登录名负责"进门",用户名负责"干活"}。先用登录名连接到SQL Server,再通过映射的用户名在具体数据库中执行操作。

\qquad 根据模板可以得到答案。

\begin{mdframed}[backgroundcolor=blue!5]
\begin{verbatim}
-- 创建登录名
create login sql_user1
with Password='nulibeikao'

-- 映射登录名到用户
create user sql_user1
for login sql_user1
\end{verbatim}
\end{mdframed}

\textbf{注意:} \textcolor{Red}{如果题目中要求建立映射关系,但前面没有建立登录名,要先建立登录名再建立映射关系。}

\subsubsection{问题二}

\textbf{题目} \emph{对象权限,授予sql\_user1用户在student表中的插入,更新,查询权利。}

\vspace{6pt}

\qquad 这道题涉及到数据库用户权限管理,我先介绍一下数据库的权限类别,如表\ref{table:2}。

\begin{table}[H]
  \centering
  \begin{tabularx}{\textwidth}{p{5cm}|X}
    \toprule
    \textbf{权限类别} & \textbf{描述} \\
    \hline
    SELECT & 查询权限,允许用户查看表或视图中的数据 \\
    INSERT & 插入权限,允许用户向表中添加新的数据行 \\
    UPDATE & 更新权限,允许用户修改表中现有的数据 \\
    DELETE & 删除权限,允许用户删除表中的数据行 \\
    REFERENCES & 引用权限,允许用户创建引用该表的外键约束 \\
    ALTER & 修改权限,允许用户修改表结构(添加/删除列等) \\
    INDEX & 索引权限,允许用户在表上创建或删除索引 \\
    EXECUTE & 执行权限,允许用户执行存储过程或函数 \\
    CREATE TABLE & 创建表权限,允许用户在数据库中创建新表 \\
    CREATE VIEW & 创建视图权限,允许用户在数据库中创建视图 \\
    CREATE PROCEDURE & 创建存储过程权限,允许用户创建存储过程 \\
    \bottomrule
  \end{tabularx}
  \caption{SQL Server权限类别说明}
  \label{table:2}
\end{table}

\qquad 下面给出授予用户权限的模板并给出几个例子。

\begin{mdframed}[backgroundcolor=gray!10]
\begin{verbatim}
-- 模板
grant 权限类别[,...n] [on 表名[,...n]] to 用户名

-- 授予用户创建数据库的权限
grant create database to Qtcyy

-- 授予用户对student表进行插入、更新数据的权限
grant insert, update on student to Qtcyy
\end{verbatim}
\end{mdframed}

\qquad 根据模板可以得到答案。

\begin{mdframed}[backgroundcolor=blue!5]
\begin{verbatim}
grant insert, update, select on student to sql_user1
\end{verbatim}
\end{mdframed}

\textbf{老师提醒:} \textcolor{Red}{注意题目的各个权限的顺序,考试系统内标准比较严,请按照顺序给。}

\subsubsection{问题三}

\textbf{题目} \emph{语句权限,允许sql\_user1用户在数据库上创建视图、存储过程的权限。}

\vspace{6pt}

\qquad 这题考查的是创建视图、存储过程的权限,根据模板可以得到答案。

\begin{mdframed}[backgroundcolor=blue!5]
\begin{verbatim}
grant create view, create procedure to sql_user1
\end{verbatim}
\end{mdframed}

\subsubsection{问题四}

\textbf{题目} \emph{语句权限,拒绝sql\_user1用户在数据库上创建表的权限}

\vspace{6pt}

\qquad 这道题涉及到拒绝用户权限,需要使用DENY语句。下面给出拒绝用户权限的模板。

\begin{mdframed}[backgroundcolor=gray!10]
\begin{verbatim}
-- 拒绝权限模板
deny 权限类别[,...n] [on 表名[,...n]] to 用户名

-- 拒绝用户创建表的权限
deny create table to Qtcyy

-- 拒绝用户对student表进行删除数据的权限
deny delete on student to Qtcyy
\end{verbatim}
\end{mdframed}

\qquad 根据模板可以得到答案。

\begin{mdframed}[backgroundcolor=blue!5]
\begin{verbatim}
deny create table to sql_user1
\end{verbatim}
\end{mdframed}

\qquad \textbf{权限控制语句对比:}

\begin{itemize}
  \item \textbf{GRANT}:授予权限,允许用户执行某些操作
  \item \textbf{DENY}:拒绝权限,明确禁止用户执行某些操作
  \item \textbf{REVOKE}:撤销权限,取消之前授予或拒绝的权限
\end{itemize}

\textbf{注意:} \textcolor{Red}{DENY的优先级高于GRANT,即使用户通过其他角色获得了权限,DENY仍然有效。}

\subsubsection{问题五}

\textbf{题目} \emph{对象权限,拒绝sql\_user1用户在sc表上删除数据的权利}

\vspace{6pt}

\qquad 根据模板可以得到答案。

\begin{mdframed}[backgroundcolor=blue!5]
\begin{verbatim}
deny delete on sc to sql_user1
\end{verbatim}
\end{mdframed}

\subsection{知识点拓展}

\subsubsection{身份验证模式}

\qquad SQL Server支持两种身份验证模式:

\begin{itemize}
  \item \textbf{Windows身份验证模式}:
    \begin{itemize}
      \item 使用Windows用户账户或Windows组账户进行身份验证
      \item 更安全,因为利用了Windows的安全机制
      \item 支持密码策略、账户锁定等Windows安全功能
      \item 适用于企业内部网络环境
    \end{itemize}

  \item \textbf{混合模式(SQL Server和Windows身份验证)}:
    \begin{itemize}
      \item 同时支持Windows身份验证和SQL Server身份验证
      \item SQL Server身份验证使用存储在SQL Server中的登录名和密码
      \item 适用于需要支持非Windows客户端的环境
      \item 默认启用sa账户(系统管理员账户)
    \end{itemize}
\end{itemize}

\subsubsection{服务器角色}

\qquad SQL Server提供了预定义的服务器级固定角色,用于管理服务器级权限:

\begin{table}[H]
  \centering
  \begin{tabularx}{\textwidth}{p{4cm}|X}
    \toprule
    \textbf{服务器角色} & \textbf{权限描述} \\
    \hline
    sysadmin & 系统管理员,拥有服务器的完全控制权限 \\
    serveradmin & 服务器管理员,可以更改服务器范围的配置选项和关闭服务器 \\
    securityadmin & 安全管理员,管理登录名和权限,可以重置密码 \\
    processadmin & 进程管理员,可以终止在SQL Server实例中运行的进程 \\
    setupadmin & 安装管理员,可以添加和删除链接服务器 \\
    bulkadmin & 批量管理员,可以执行BULK INSERT语句 \\
    diskadmin & 磁盘管理员,管理磁盘文件 \\
    dbcreator & 数据库创建者,可以创建、修改、删除和还原任何数据库 \\
    public & 公共角色,每个SQL Server登录名都属于此角色 \\
    \bottomrule
  \end{tabularx}
  \caption{SQL Server服务器角色}
  \label{table:3}
\end{table}

\qquad \textbf{添加用户到服务器角色的语法:}

\begin{mdframed}[backgroundcolor=gray!10]
\begin{verbatim}
-- 将登录名添加到服务器角色
ALTER SERVER ROLE 角色名 ADD MEMBER 登录名

-- 示例:将sql_user1添加到dbcreator角色
ALTER SERVER ROLE dbcreator ADD MEMBER sql_user1
\end{verbatim}
\end{mdframed}

\subsubsection{数据库角色}

\qquad 数据库角色用于管理数据库级别的权限,包括固定数据库角色和用户定义角色:

\begin{table}[H]
  \centering
  \begin{tabularx}{\textwidth}{p{4cm}|X}
    \toprule
    \textbf{数据库角色} & \textbf{权限描述} \\
    \hline
    db\_owner & 数据库所有者,拥有数据库的完全控制权限 \\
    db\_accessadmin & 访问管理员,可以添加或删除数据库用户 \\
    db\_securityadmin & 安全管理员,可以管理角色成员身份和权限 \\
    db\_ddladmin & DDL管理员,可以运行任何DDL命令(CREATE、ALTER、DROP) \\
    db\_backupoperator & 备份操作员,可以备份数据库 \\
    db\_datareader & 数据读取者,可以从所有用户表中读取数据 \\
    db\_datawriter & 数据写入者,可以在所有用户表中添加、更新或删除数据 \\
    db\_denydatareader & 拒绝数据读取者,不能读取数据库中的任何数据 \\
    db\_denydatawriter & 拒绝数据写入者,不能修改数据库中的任何数据 \\
    public & 公共角色,每个数据库用户都属于此角色 \\
    \bottomrule
  \end{tabularx}
  \caption{SQL Server数据库角色}
  \label{table:4}
\end{table}

\qquad \textbf{数据库角色管理语法:}

\begin{mdframed}[backgroundcolor=gray!10]
\begin{verbatim}
-- 将用户添加到数据库角色
ALTER ROLE 角色名 ADD MEMBER 用户名

-- 从数据库角色中删除用户
ALTER ROLE 角色名 DROP MEMBER 用户名

-- 创建用户定义的数据库角色
CREATE ROLE 角色名

-- 示例:将sql_user1添加到db_datareader角色
ALTER ROLE db_datareader ADD MEMBER sql_user1
\end{verbatim}
\end{mdframed}

\qquad \textbf{权限继承关系:}

\begin{itemize}
  \item 用户可以同时属于多个角色
  \item 用户获得所属所有角色的权限(权限的并集)
  \item DENY权限始终优先于GRANT权限
  \item 角色权限与直接授予用户的权限相互补充
\end{itemize}