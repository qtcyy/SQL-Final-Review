\section{第五题:游标的创建与使用}

\subsection{题目}

\qquad 将student表中的同学按照姓名升序后的前7位同学的相关信息打印在消息窗格中,格式为:姓名+‘的年龄为’+年龄。
游标名为stu\_cur,为了方便,将姓名、年龄存储在局部变量为@stu\_name, @stu\_age上。 局部变量声明在打开游标之前。

\subsection{解析}

\qquad 这道题需要创建并打开游标,所以我先介绍一下游标的创建。

\subsubsection{创建游标}

\qquad 创建游标的简单模板

\begin{mdframed}[backgroundcolor=gray!10]
\begin{verbatim}
declare 游标名 cursor
for 数据库查询语句
\end{verbatim}
\end{mdframed}

\qquad \textbf{说明:} \emph{这不是完整的游标创建模板,完整版考试不要求。}

\qquad 简单示例

\begin{mdframed}[backgroundcolor=gray!10]
\begin{verbatim}
declare Mycur cursor
for select Sno,Sname
from student
where Ssex='男'
\end{verbatim}
\end{mdframed}

\subsubsection{打开游标}

\qquad 打开游标的模板

\begin{mdframed}[backgroundcolor=gray!10]
\begin{verbatim}
open [local | global] 游标名 | 游标变量名
\end{verbatim}
\end{mdframed}

\qquad 简单示例

\begin{mdframed}[backgroundcolor=gray!10]
\begin{verbatim}
open Mycur
\end{verbatim}
\end{mdframed}

\subsubsection{读取游标}

\qquad 游标的读取使用FETCH语句,其过程分两步:

\begin{enumerate}
  \item 将游标当前指向的记录保存到一个局部变量中
  \item 游标自动移动到下一条记录
\end{enumerate}

\qquad 当记录读入局部变量后,就可以根据需要进行处理。

\qquad 读取游标的模板

\begin{mdframed}[backgroundcolor=gray!10]
\begin{verbatim}
fetch [[next | prior | first | last |
absolute{n | @nvar | relative{n | @nvar}}]
from ] 游标名 [into @局部变量名 [,...n]]
\end{verbatim}
\end{mdframed}

\qquad 参数含义如\ref{table:cursor:1}。

\begin{table}[h]
  \centering
  \begin{tabular}{|l|l|}
    \hline
    \textbf{参数} & \textbf{含义} \\
    \hline
    NEXT & 移动到下一条记录(默认选项) \\
    \hline
    PRIOR & 移动到上一条记录 \\
    \hline
    FIRST & 移动到第一条记录 \\
    \hline
    LAST & 移动到最后一条记录 \\
    \hline
    ABSOLUTE n & 移动到第n条记录(正数从头开始,负数从尾开始) \\
    \hline
    RELATIVE n & 相对当前位置移动n条记录(正数向前,负数向后) \\
    \hline
    INTO @变量名 & 将获取的数据存储到指定的局部变量中 \\
    \hline
  \end{tabular}
  \caption{FETCH语句参数含义}
  \label{table:cursor:1}
\end{table}

\subsubsection{题目解答}

\qquad 根据题目要求,需要创建游标并获取前7位学生信息。完整代码如下:

\begin{mdframed}[backgroundcolor=blue!5]
\begin{verbatim}
-- 声明局部变量(在打开游标之前)
declare @stu_name varchar(20);
declare @stu_age int;
declare @count int = 0;

-- 声明游标
declare stu_cur cursor
for select sname, sage
    from student
    order by sname asc;

-- 打开游标
open stu_cur;

-- 读取第一条记录
fetch next from stu_cur into @stu_name, @stu_age;

-- 循环处理前7条记录
while @@fetch_status = 0 and @count < 7
begin
    -- 打印信息到消息窗格
    print @stu_name + '的年龄为' + cast(@stu_age as varchar(10));

    -- 计数器加1
    set @count = @count + 1;

    -- 读取下一条记录
    fetch next from stu_cur into @stu_name, @stu_age;
end

-- 关闭游标
close stu_cur;

-- 释放游标
deallocate stu_cur;
\end{verbatim}
\end{mdframed}

\textbf{代码说明:}
\begin{itemize}
  \item \textbf{@@FETCH\_STATUS}:系统变量,表示上一次FETCH操作的状态(0表示成功)
  \item \textbf{CAST函数}:将整数类型的年龄转换为字符串,便于拼接
  \item \textbf{计数器@count}:确保只处理前7条记录
  \item \textbf{CLOSE}:关闭游标,释放资源
  \item \textbf{DEALLOCATE}:释放游标内存
\end{itemize}
