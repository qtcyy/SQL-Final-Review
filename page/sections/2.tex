\section{第二题:视图与索引}

\subsection{题目}

\qquad 视图与索引

\begin{enumerate}
  \item 创建带加密选课视图view\_sc,列名依次显示为 学号,姓名,年龄
  \item 在学生表中的sname上创建非聚集索引(Sname\_ind)
  \item 在给课程名创建唯一非聚集索引(Cname\_ind)
\end{enumerate}

\subsection{解析}
\qquad 三个问题,分步进行解析。

\subsubsection{第一问}

\textbf{题目} \emph{创建带加密选课视图view\_sc,列名依次显示为 学号,姓名,年龄。}

\vspace{6pt}

\qquad 分析一下创建视图的模板:

\begin{mdframed}[backgroundcolor=gray!10]
\begin{verbatim}
create view [视图名称]([列名显示:name1,name2,...])
/*列名显示部分设置的列名一一对应select选出来的列*/
[with encryption] /*如果要求创建带加密的视图,需要添加这一句*/
as
/*下面为需要展示的内容的select语句*/
select [col1,col2,...]
...
\end{verbatim}
\end{mdframed}

\qquad 根据创建视图的模板可以得到答案:

\begin{mdframed}[backgroundcolor=blue!5]
\begin{verbatim}
create view view_sc(学号,姓名,年龄)
with encryption
as
select sno,sname,sage
from student
\end{verbatim}
\end{mdframed}

\subsubsection{第二问}

\textbf{题目} \emph{在学生表中的sname上创建非聚集索引(Sname\_ind)}

\qquad 分析一下创建索引的模板:

\begin{mdframed}[backgroundcolor=gray!10]
  \label{index:1}
\begin{verbatim}
create [unique 唯一索引] [clustered 聚集索引| nonclustered 非聚集索引]
/*如果不填默认为非聚集索引 nonclustered*/
index [填索引名称]
on [表名|视图名](这个表或视图中的列名:[col1,col2,...])
\end{verbatim}
\end{mdframed}

\qquad 根据模板可以得到答案:

\begin{mdframed}[backgroundcolor=blue!5]
\begin{verbatim}
create index Sname_ind
on Student(sname)
\end{verbatim}
\end{mdframed}

\subsubsection{第三问}

\textbf{题目} \emph{在给课程名创建唯一非聚集索引(Cname\_ind)}

\qquad 根据创建索引的模板\ref{index:1}可以得到答案:

\begin{mdframed}[backgroundcolor=blue!5]
\begin{verbatim}
create unique nonclustered /*题目要求唯一,需要加unique*/
index Cname_ind
on Course(Cname)
\end{verbatim}
\end{mdframed}

\subsection{知识点拓展}

\subsubsection{索引类型及特征}

\paragraph{聚集索引(Clustered Index)}
\begin{itemize}
  \item \textbf{特征}:数据行的物理存储顺序与索引键值的逻辑顺序相同
  \item \textbf{限制}:每个表只能有一个聚集索引
  \item \textbf{作用}:提高范围查询和排序操作的效率
  \item \textbf{适用场景}:经常进行范围查询的列,如主键、日期列
\end{itemize}

\paragraph{非聚集索引(Non-clustered Index)}
\begin{itemize}
  \item \textbf{特征}:索引的逻辑顺序与数据行的物理存储顺序无关
  \item \textbf{限制}:每个表可以有多个非聚集索引(最多999个)
  \item \textbf{作用}:提高特定列的查询速度,但不改变数据的物理存储
  \item \textbf{适用场景}:经常用于WHERE子句、JOIN条件的列
\end{itemize}

\paragraph{唯一索引(Unique Index)}
\begin{itemize}
  \item \textbf{特征}:确保索引键值的唯一性,不允许重复值
  \item \textbf{作用}:既提高查询性能,又保证数据完整性
  \item \textbf{适用场景}:需要保证唯一性的列,如身份证号、学号等
\end{itemize}

\paragraph{复合索引(Composite Index)}
\begin{itemize}
  \item \textbf{特征}:基于多个列创建的索引
  \item \textbf{作用}:提高多列组合查询的效率
  \item \textbf{注意}:遵循"最左前缀"原则,索引列的顺序很重要
\end{itemize}

\subsubsection{索引使用原则}

\begin{enumerate}
  \item \textbf{选择性原则}:在选择性高(重复值少)的列上创建索引效果更好
  \item \textbf{频率原则}:在经常用于查询条件的列上创建索引
  \item \textbf{维护成本}:索引会增加INSERT、UPDATE、DELETE操作的开销
  \item \textbf{存储空间}:索引需要额外的存储空间
\end{enumerate}

\begin{mdframed}[backgroundcolor=yellow!10]
  \textbf{小贴士:}索引是一把双刃剑,能够显著提高查询性能,但也会增加数据修改的开销和存储空间的占用。因此需要根据实际的查询需求和数据更新频率来合理设计索引策略。
\end{mdframed}

\subsubsection{索引的管理方式}

\textbf{1. 删除索引}

\begin{mdframed}[backgroundcolor=gray!10]
\begin{verbatim}
drop index [表名.索引名 | 视图名.索引名]

/*一个简单的例子:删除第二问中的Sname_ind索引*/
drop index Student.Sname_ind
\end{verbatim}
\end{mdframed}

\textbf{注意:}\texttt{表在创建时如果对某一列进行了主键约束会自动自动创建一个索引,这个索引是无法用drop index进行删除的。}

\vspace{6pt}

\textbf{2. 查看索引}

\qquad 可以使用SSMS软件进行查看,也可以用代码进行查看。这里介绍一下代码怎么查看。

\begin{mdframed}[backgroundcolor=gray!10]
\begin{verbatim}
use [数据库名]
go
sp_helpindex [表名]
go
\end{verbatim}
\end{mdframed}

\subsubsection{重命名索引}

\qquad 利用sp\_rename可以重命名索引

\begin{mdframed}[backgroundcolor=gray!10]
\begin{verbatim}
use [数据库名]
go
sp_rename '表名.原索引名', '新索引名'
go

/*举一个例子*/
use 学生选课
go
sp_rename 'student.Sname_index', 'index_Sname'
go
\end{verbatim}
\end{mdframed}