\section{第六题:存储过程}

\subsection{题目}

\begin{enumerate}
\item 创建存储过程(p\_stu),实现给定学号(局部变量名为@stu\_sno, 并取默认值为95001),列出年龄大于该同学的学生信息,姓名和年龄(列名不变)。存储过程中的查询语句通过子查询实现。
\item 执行上述存储过程,取95004实验。
\item 创建存储过程(p\_stu2),实现给定学号(局部变量名为@stu\_sno, 并取默认值为95001),列出与该学生属于同一系的其他学生姓名和年龄(原样显示)。存储过程中的查询语句通过自身链接(别名采用s1,s2,且通过s2表返回)实现。
\item 通过95002验证。
\item 创建存储过程p\_count\_cs,根据输入学生学号(@stu\_sno),返回该学生选了多少门课。返回值为整数型,且取名为@c\_ss。
\item 检查学号为95001的情况验证准确性。输出变量取名为@pp\_cnos,打印内容为仅@pp\_cnos。考虑如果是指定课程号,返回有多少学生选了该课程,应该怎么写和验证。
\end{enumerate}

\subsection{解析}

\qquad 在解析题目之前,我先讲一下存储过程的创建和执行。

\subsubsection{存储过程的创建}

\qquad 带参数的存储过程创建简单模板

\begin{mdframed}[backgroundcolor=gray!10]
\begin{verbatim}
create procedure 存储过程名
[{@参数名称 参数数据类型} [ = 参数的默认值]
[output] ] -- 参数后面带output的为输出参数
[,...n]
as
sql_statement
\end{verbatim}
\end{mdframed}

\qquad 简单示例

\begin{mdframed}[backgroundcolor=gray!10]
\begin{verbatim}
-- 带输入参数的存储过程
create procedure p_student
@Sno char(5)
as
select Sname, Sdept
from student where Sno=@Sno

-- 带输出参数的存储过程
create procedure p_sum
@var1 int, @var2 int,
@var3 int output -- 输出参数
as
set @var3 = @var1 * @var2
\end{verbatim}
\end{mdframed}

\qquad 不带参数的存储过程创建简单模板

\begin{mdframed}[backgroundcolor=gray!10]
\begin{verbatim}
create procedure 存储过程名
as
sql_statement
\end{verbatim}
\end{mdframed}

\qquad \textbf{说明:} \emph{完整版模板考试不要求,
\textcolor{Red}{变量前必须加@}。}

\qquad 简单示例

\begin{mdframed}[backgroundcolor=gray!10]
\begin{verbatim}
create procedure p_course
as
select * from course
\end{verbatim}
\end{mdframed}

\subsubsection{存储过程的执行}

\qquad 不带参数的存储过程执行模板

\begin{mdframed}[backgroundcolor=gray!10]
\begin{verbatim}
exec 存储过程名
\end{verbatim}
\end{mdframed}

\qquad 简单示例

\begin{mdframed}[backgroundcolor=gray!10]
\begin{verbatim}
exec p_course
\end{verbatim}
\end{mdframed}

\qquad 带输入参数的存储过程

\begin{mdframed}[backgroundcolor=gray!10]
\begin{verbatim}
exec 存储过程名
[@参数名 = 参数值][default]
[,...n]
\end{verbatim}
\end{mdframed}

\qquad 简单示例

\begin{mdframed}[backgroundcolor=gray!10]
\begin{verbatim}
exec p_grade2 @dept='计算机系'
\end{verbatim}
\end{mdframed}

\qquad 带输出参数的存储过程

\begin{mdframed}[backgroundcolor=gray!10]
\begin{verbatim}
exec 存储过程名
[[@参数名=]{参数值 | @变量[output] | [默认值]}][,...n]
\end{verbatim}
\end{mdframed}

\qquad 简单示例

\begin{mdframed}[backgroundcolor=gray!10]
\begin{verbatim}
declare @res int
exec p_sum @var1=3,@var2=8,@res=@res output

-- 或者使用参数顺序调用
exec p_sum 3,8,@res output
\end{verbatim}
\end{mdframed}

\subsubsection{问题一}

\textbf{题目} \emph{创建存储过程(p\_stu),实现给定学号(局部变量名为@stu\_sno, 并取默认值为95001),列出年龄大于该同学的学生信息,姓名和年龄(列名不变)。存储过程中的查询语句通过子查询实现。}

\qquad 根据存储过程的知识可得到答案。

\begin{mdframed}[backgroundcolor=blue!5]
\begin{verbatim}
create procedure p_stu
@stu_sno char(5) = '95001'
as
select Sname, Sage
from student
where Sage > (
    select Sage from student
    where Sno=@stu_sno
)
\end{verbatim}
\end{mdframed}

\subsubsection{问题二}

\textbf{题目} \emph{执行上述存储过程,取95004实验。}

\vspace{6pt}

\qquad 传入95004作为参数执问题一的存储过程,得到答案。

\begin{mdframed}[backgroundcolor=blue!5]
\begin{verbatim}
exec p_stu '95004' -- 注意char类型需要加引号
\end{verbatim}
\end{mdframed}

\subsubsection{问题三}

\textbf{题目} \emph{创建存储过程(p\_stu2),实现给定学号(局部变量名为@stu\_sno, 并取默认值为95001),列出与该学生属于同一系的其他学生姓名和年龄(原样显示)。存储过程中的查询语句通过自身链接(别名采用s1,s2,且通过s2表返回)实现。}

\vspace{6pt}

\qquad 难点在于如何查询。得到答案。

\begin{mdframed}[backgroundcolor=blue!5]
\begin{verbatim}
create procedure p_stu2
@stu_sno char(5) = '95001'
as
select s2.Sname, s2.Sage
from student s2 join student s1
on s2.Sdept=s1.Sdept
where s1.Sno=@stu_sno and s2.Sno != @stu_sno
\end{verbatim}
\end{mdframed}

\qquad \textbf{注意:} \textcolor{red}{本题要查询的是“其他学生”,不包括输入的学生,注意排除。}

\subsubsection{问题四}

\textbf{题目} \emph{通过95002验证。}

\vspace{6pt}

\qquad 执行时传入参数即可得到答案。

\begin{mdframed}[backgroundcolor=blue!5]
\begin{verbatim}
exec p_stu2 '95002'
\end{verbatim}
\end{mdframed}

\subsubsection{问题五}

\textbf{题目} \emph{创建存储过程p\_count\_cs,根据输入学生学号(@stu\_sno),返回该学生选了多少门课。返回值为整数型,且取名为@c\_ss。}

\vspace{6pt}

\qquad 这道题的考点在于存储过程的输出参数。得到答案。

\begin{mdframed}[backgroundcolor=blue!5]
\begin{verbatim}
create procedure p_count_cs
@stu_sno char(5), @c_ss int output
as
select @c_ss=count(*)
from SC
where Sno=@stu_sno
\end{verbatim}
\end{mdframed}

\subsubsection{问题六}

\textbf{题目} \emph{检查学号为95001的情况验证准确性。输出变量取名为@pp\_cnos,打印内容为仅@pp\_cnos。考虑如果是指定课程号,返回有多少学生选了该课程,应该怎么写和验证。}

\vspace{6pt}

\qquad 这题的考点是带参数的存储过程的执行。得到答案。

\begin{mdframed}[backgroundcolor=blue!5]
\begin{verbatim}
-- 定义变量
declare @pp_cnos int

-- 执行存储过程
exec p_count_cs '95001', @pp_cnos output

-- 打印结果
print @pp_cnos
\end{verbatim}
\end{mdframed}

\textbf{注意:} \textcolor{red}{输出参数的后面要加output,不要忘记。}

\subsection{知识点拓展}

\subsubsection{查看存储过程}

\qquad 可以通过以下几种方式查看存储过程:

\begin{enumerate}
\item \textbf{查看所有存储过程}:
    \begin{verbatim}
    SELECT * FROM sys.procedures;
    \end{verbatim}

\item \textbf{查看存储过程定义}:
    \begin{verbatim}
    EXEC sp_helptext '存储过程名';
    \end{verbatim}

\item \textbf{查看存储过程参数}:
    \begin{verbatim}
    EXEC sp_help '存储过程名';
    \end{verbatim}
\end{enumerate}

\textbf{示例:}
\begin{verbatim}
EXEC sp_helptext 'p_stu';
\end{verbatim}

\subsubsection{删除用户存储过程}

\qquad 删除存储过程的语法:

\begin{mdframed}[backgroundcolor=gray!10]
\begin{verbatim}
DROP PROCEDURE 存储过程名;
\end{verbatim}
\end{mdframed}

\textbf{示例:}
\begin{verbatim}
-- 删除单个存储过程
DROP PROCEDURE p_stu;

-- 删除多个存储过程
DROP PROCEDURE p_stu, p_stu2, p_count_cs;
\end{verbatim}

\textbf{注意:}删除存储过程是不可逆操作,请谨慎执行。

\subsubsection{修改存储过程}

\qquad 修改存储过程的语法:

\begin{mdframed}[backgroundcolor=gray!10]
\begin{verbatim}
ALTER PROCEDURE 存储过程名
[参数列表]
AS
BEGIN
    -- 修改后的SQL语句
END
\end{verbatim}
\end{mdframed}

\textbf{示例:}
\begin{lstlisting}[backgroundcolor=\color{Gray!10}]
ALTER PROCEDURE p_stu
@stu_sno CHAR(5) = '95001'
AS
BEGIN
    SELECT Sname, Sage, Sdept  -- 增加系别信息
    FROM student
    WHERE Sage > (
        SELECT Sage FROM student
        WHERE Sno = @stu_sno
    )
END
\end{lstlisting}

\begin{mdframed}[backgroundcolor=yellow!10]
\textbf{存储过程管理要点:}
\begin{itemize}
\item 修改存储过程时使用ALTER PROCEDURE,而不是重新CREATE
\item 存储过程名在数据库中必须唯一
\item 定期检查和优化存储过程的性能
\item 为存储过程添加适当的注释,便于维护
\item 测试存储过程的各种参数组合,确保逻辑正确
\end{itemize}
\end{mdframed}