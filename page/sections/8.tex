\section{第八题:数据库的备份与恢复}

\subsection{题目}

\begin{enumerate}
  \item 创建销售管理数据库的完整备份,备份设备名为abc+学号后3位,如:abc109(备份设备处截图),位置放置在D:\textbackslash abc。
  \item 插入一个新员工(自己的名字),然后进行一次差异备份。
  \item 删除刚添加的新员工信息,再进行一次日志备份。
  \item 恢复完整备份。
  \item 恢复完全备份+差异备份。(或者换一种问法,即,恢复到刚刚插入新员工的位置)
  \item 恢复完全备份+差异备份+日志备份。(或者换一种问法,即,恢复到刚刚删除新员工的位置)
\end{enumerate}

\subsection{解析}
\qquad 这个题涉及数据库的备份和恢复备份,我会介绍一下数据库备份与恢复相关的知识点。

\subsubsection{创建备份设备}

\qquad 备份设备是SQL Server中用于存储备份文件的逻辑设备,可以是磁盘文件或磁带设备。通过创建备份设备,可以简化备份操作,避免每次都指定完整的文件路径。

\qquad \textbf{模板:}
\begin{mdframed}[backgroundcolor=gray!10]
\begin{verbatim}
-- 创建备份设备模板
EXEC sp_addumpdevice
    @devtype = '[设备类型:disk/tape]',
    @logicalname = '[备份设备逻辑名称]',
    @physicalname = '[备份文件完整路径]'
\end{verbatim}
\end{mdframed}

\qquad \textbf{示例:}
\begin{mdframed}[backgroundcolor=blue!5]
\begin{verbatim}
-- 为销售管理数据库创建备份设备abc109
EXEC sp_addumpdevice
    @devtype = 'disk',
    @logicalname = 'abc109',
    @physicalname = 'D:\abc\abc109.bak'
\end{verbatim}
\end{mdframed}

\subsubsection{备份类型}

\textbf{1. 完整备份}

\qquad \textbf{概念说明:}完整备份会备份整个数据库,包括所有数据页、索引页、系统表和数据库结构信息。它是所有其他备份类型的基础。

\qquad \textbf{特点:}
\begin{itemize}
  \item \textbf{优点}:包含完整的数据库信息,恢复时不依赖其他备份文件
  \item \textbf{缺点}:备份时间长,占用存储空间大
  \item \textbf{适用场景}:定期的基础备份,作为其他备份类型的基础
  \item \textbf{恢复能力}:可以独立恢复到备份时的状态
\end{itemize}

\qquad \textbf{模板:}
\begin{mdframed}[backgroundcolor=gray!10]
\begin{verbatim}
-- 完整备份模板
BACKUP DATABASE [数据库名称]
TO [备份设备名称或DISK='文件路径']
WITH FORMAT, INIT,
NAME = '[备份名称描述]',
SKIP, NOREWIND, NOUNLOAD, STATS = 10
\end{verbatim}
\end{mdframed}

\qquad \textbf{示例:}
\begin{mdframed}[backgroundcolor=blue!5]
\begin{verbatim}
-- 销售管理数据库完整备份
BACKUP DATABASE 销售管理 TO abc109
WITH FORMAT, INIT,
NAME = '销售管理-完整数据库备份',
SKIP, NOREWIND, NOUNLOAD, STATS = 10
\end{verbatim}
\end{mdframed}

\textbf{2. 差异备份}

\qquad \textbf{概念说明:}差异备份只备份自上次完整备份以来发生变化的数据页。它基于数据库的差异位图(Differential Bitmap)来识别哪些页面发生了变化。

\qquad \textbf{特点:}
\begin{itemize}
  \item \textbf{优点}:备份速度快,占用空间小,备份时间可预测
  \item \textbf{缺点}:恢复时需要完整备份和最新的差异备份,随时间推移备份会越来越大
  \item \textbf{适用场景}:数据变化频繁但希望快速备份的环境
  \item \textbf{恢复能力}:需要与完整备份配合使用
\end{itemize}

\qquad \textbf{模板:}
\begin{mdframed}[backgroundcolor=gray!10]
\begin{verbatim}
-- 差异备份模板
BACKUP DATABASE [数据库名称]
TO [备份设备名称或DISK='文件路径']
WITH DIFFERENTIAL,
NAME = '[备份名称描述]',
SKIP, NOREWIND, NOUNLOAD, STATS = 10
\end{verbatim}
\end{mdframed}

\qquad \textbf{示例:}
\begin{mdframed}[backgroundcolor=blue!5]
\begin{verbatim}
-- 插入新员工后进行差异备份
INSERT INTO 员工表 VALUES ('张三', '开发部', 8000)

BACKUP DATABASE 销售管理
TO DISK = 'D:\abc\销售管理_差异备份.bak'
WITH DIFFERENTIAL,
NAME = '销售管理-差异数据库备份',
SKIP, NOREWIND, NOUNLOAD, STATS = 10
\end{verbatim}
\end{mdframed}

\textbf{3. 事务日志备份}

\qquad \textbf{概念说明:}事务日志备份用于备份事务日志文件中的活动事务记录,记录自上次日志备份以来的所有数据库活动。它是实现时点恢复的关键。

\qquad \textbf{特点:}
\begin{itemize}
  \item \textbf{优点}:备份速度最快,文件最小,支持时点恢复,数据丢失最少
  \item \textbf{缺点}:需要数据库处于完整恢复模式,必须定期执行以防日志文件过大
  \item \textbf{适用场景}:对数据安全要求极高,需要频繁备份的环境
  \item \textbf{恢复能力}:可以恢复到任意时间点,最大限度减少数据丢失
\end{itemize}

\qquad \textbf{模板:}
\begin{mdframed}[backgroundcolor=gray!10]
\begin{verbatim}
-- 事务日志备份模板(需要完整恢复模式)
ALTER DATABASE [数据库名称] SET RECOVERY FULL

BACKUP LOG [数据库名称]
TO [备份设备名称或DISK='文件路径']
WITH NAME = '[备份名称描述]',
SKIP, NOREWIND, NOUNLOAD, STATS = 10
\end{verbatim}
\end{mdframed}

\qquad \textbf{示例:}
\begin{mdframed}[backgroundcolor=blue!5]
\begin{verbatim}
-- 删除员工后进行日志备份
DELETE FROM 员工表 WHERE 姓名 = '张三'

-- 设置完整恢复模式
ALTER DATABASE 销售管理 SET RECOVERY FULL

-- 进行日志备份
BACKUP LOG 销售管理
TO DISK = 'D:\abc\销售管理_日志备份.trn'
WITH NAME = '销售管理-事务日志备份',
SKIP, NOREWIND, NOUNLOAD, STATS = 10
\end{verbatim}
\end{mdframed}

\subsubsection{数据库恢复}

\textbf{1. 恢复完整备份}

\qquad \textbf{说明:}这是最简单的恢复方式,直接从完整备份恢复整个数据库。会丢失备份时间点之后的所有数据变更。

\qquad \textbf{模板:}
\begin{mdframed}[backgroundcolor=gray!10]
\begin{verbatim}
-- 恢复完整备份模板
RESTORE DATABASE [数据库名称]
FROM [备份设备名称或DISK='文件路径']
WITH REPLACE, RECOVERY
\end{verbatim}
\end{mdframed}

\qquad \textbf{示例:}
\begin{mdframed}[backgroundcolor=blue!5]
\begin{verbatim}
-- 方式1:从备份设备恢复
RESTORE DATABASE 销售管理 FROM abc109
WITH REPLACE, RECOVERY

-- 方式2:从文件路径恢复
RESTORE DATABASE 销售管理
FROM DISK = 'D:\abc\abc109.bak'
WITH REPLACE, RECOVERY
\end{verbatim}
\end{mdframed}

\textbf{2. 恢复完整备份+差异备份}

\qquad \textbf{说明:}先恢复完整备份,再应用差异备份。可以恢复到差异备份时的状态,减少了部分数据丢失。注意必须使用NORECOVERY选项,直到最后一步才使用RECOVERY。
这在多步骤恢复备份过程中都适用。

\qquad \textbf{模板:}
\begin{mdframed}[backgroundcolor=gray!10]
\begin{verbatim}
-- 恢复完整备份+差异备份模板
RESTORE DATABASE [数据库名称]
FROM [完整备份设备或路径]
WITH REPLACE, NORECOVERY

RESTORE DATABASE [数据库名称]
FROM [差异备份设备或路径]
WITH RECOVERY
\end{verbatim}
\end{mdframed}

\begin{mdframed}[backgroundcolor=yellow!20]
  \textbf{NORECOVERY与RECOVERY选项的重要性:}
  \begin{itemize}
    \item \textbf{NORECOVERY}:告诉SQL Server这不是恢复过程的最后一步,数据库保持在"正在恢复"状态,允许应用后续的备份文件。此时数据库无法访问。
    \item \textbf{RECOVERY}:告诉SQL Server这是恢复过程的最后一步,完成恢复过程并使数据库重新联机,用户可以正常访问。
    \item \textbf{为什么这样设计}:如果在中间步骤使用RECOVERY,数据库会立即联机,导致无法继续应用后续的备份文件,恢复过程被中断。
    \item \textbf{事务日志链}:多步骤恢复需要保持事务日志的连续性,NORECOVERY确保日志序列号(LSN)的连续性不被破坏。
  \end{itemize}
\end{mdframed}

\qquad \textbf{示例:}
\begin{mdframed}[backgroundcolor=blue!5]
\begin{verbatim}
-- 恢复到插入新员工后的状态
-- 方式1:从备份设备恢复完整备份
RESTORE DATABASE 销售管理 FROM abc109
WITH REPLACE, NORECOVERY

RESTORE DATABASE 销售管理
FROM DISK = 'D:\abc\销售管理_差异备份.bak'
WITH RECOVERY

-- 方式2:完全使用文件路径
RESTORE DATABASE 销售管理
FROM DISK = 'D:\abc\abc109.bak'
WITH REPLACE, NORECOVERY

RESTORE DATABASE 销售管理
FROM DISK = 'D:\abc\销售管理_差异备份.bak'
WITH RECOVERY
\end{verbatim}
\end{mdframed}

\textbf{3. 恢复完整备份+差异备份+日志备份}

\qquad \textbf{说明:}这是最完整的恢复策略,可以恢复到日志备份时的精确状态。通过应用事务日志,可以实现几乎零数据丢失的恢复。

\qquad \textbf{模板:}
\begin{mdframed}[backgroundcolor=gray!10]
\begin{verbatim}
-- 恢复完整+差异+日志备份模板
RESTORE DATABASE [数据库名称]
FROM [完整备份设备或路径]
WITH REPLACE, NORECOVERY

RESTORE DATABASE [数据库名称]
FROM [差异备份设备或路径]
WITH NORECOVERY

RESTORE LOG [数据库名称]
FROM [日志备份设备或路径]
WITH RECOVERY
\end{verbatim}
\end{mdframed}

\qquad \textbf{示例:}
\begin{mdframed}[backgroundcolor=blue!5]
\begin{verbatim}
-- 恢复到删除员工后的状态
-- 方式1:混合使用备份设备和文件路径
RESTORE DATABASE 销售管理 FROM abc109
WITH REPLACE, NORECOVERY

RESTORE DATABASE 销售管理
FROM DISK = 'D:\abc\销售管理_差异备份.bak'
WITH NORECOVERY

RESTORE LOG 销售管理
FROM DISK = 'D:\abc\销售管理_日志备份.trn'
WITH RECOVERY

-- 方式2:完全使用文件路径
RESTORE DATABASE 销售管理
FROM DISK = 'D:\abc\abc109.bak'
WITH REPLACE, NORECOVERY

RESTORE DATABASE 销售管理
FROM DISK = 'D:\abc\销售管理_差异备份.bak'
WITH NORECOVERY

RESTORE LOG 销售管理
FROM DISK = 'D:\abc\销售管理_日志备份.trn'
WITH RECOVERY
\end{verbatim}
\end{mdframed}

\subsubsection{问题一}

\textbf{题目} \emph{创建销售管理数据库的完整备份,备份设备名为abc+学号后3位,如:abc109(备份设备处截图),位置放置在D:\textbackslash abc。}

\qquad 这个问题需要先创建备份设备,然后进行完整备份。备份设备的作用是为备份文件创建一个逻辑名称,简化后续的备份和恢复操作。

\qquad 根据前面的创建备份设备和完整备份模板可得到答案:

\begin{mdframed}[backgroundcolor=blue!5]
\begin{verbatim}
-- 步骤1:创建备份设备abc109
EXEC sp_addumpdevice
    @devtype = 'disk',
    @logicalname = 'abc109',
    @physicalname = 'D:\abc\abc109.bak'

-- 步骤2:执行销售管理数据库的完整备份
BACKUP DATABASE 销售管理 TO abc109
WITH FORMAT, INIT,
NAME = '销售管理数据库完整备份',
SKIP, NOREWIND, NOUNLOAD, STATS = 10
\end{verbatim}
\end{mdframed}

\subsubsection{问题二}

\textbf{题目} \emph{插入一个新员工(自己的名字),然后进行一次差异备份。}

\qquad 差异备份只备份自上次完整备份以来发生变化的数据页。在插入新员工后进行差异备份,可以快速保存这个新增的数据变更。

\qquad 根据前面的差异备份模板可得到答案:

\begin{mdframed}[backgroundcolor=blue!5]
\begin{verbatim}
-- 步骤1:插入新员工(假设员工表包含姓名、部门、工资字段)
INSERT INTO 员工表 (姓名, 部门, 工资)
VALUES ('张三', '开发部', 8000)

-- 步骤2:执行差异备份
BACKUP DATABASE 销售管理
TO DISK = 'D:\abc\销售管理_差异备份.bak'
WITH DIFFERENTIAL,
NAME = '销售管理数据库差异备份-插入新员工后',
SKIP, NOREWIND, NOUNLOAD, STATS = 10
\end{verbatim}
\end{mdframed}

\subsubsection{问题三}

\textbf{题目} \emph{删除刚添加的新员工信息,再进行一次日志备份。}

\qquad 日志备份用于备份事务日志中的活动记录,记录删除操作等数据库变更。执行日志备份前需要确保数据库处于完整恢复模式。

\qquad 根据前面的事务日志备份模板可得到答案:

\begin{mdframed}[backgroundcolor=blue!5]
\begin{verbatim}
-- 步骤1:删除刚添加的员工
DELETE FROM 员工表 WHERE 姓名 = '张三'

-- 步骤2:设置数据库为完整恢复模式(如果尚未设置)
ALTER DATABASE 销售管理 SET RECOVERY FULL

-- 步骤3:执行日志备份
BACKUP LOG 销售管理
TO DISK = 'D:\abc\销售管理_日志备份.trn'
WITH NAME = '销售管理数据库日志备份-删除员工后',
SKIP, NOREWIND, NOUNLOAD, STATS = 10
\end{verbatim}
\end{mdframed}

\subsubsection{问题四}

\textbf{题目} \emph{恢复完整备份。}

\qquad 这是最基本的恢复操作,直接从完整备份恢复整个数据库到备份时的状态。

\qquad 根据前面的恢复完整备份模板可得到答案:

\begin{mdframed}[backgroundcolor=blue!5]
\begin{verbatim}
-- 方式1:从备份设备恢复
RESTORE DATABASE 销售管理 FROM abc109
WITH REPLACE, RECOVERY

-- 方式2:从文件路径恢复
RESTORE DATABASE 销售管理
FROM DISK = 'D:\abc\abc109.bak'
WITH REPLACE, RECOVERY
\end{verbatim}
\end{mdframed}

\subsubsection{问题五}

\textbf{题目} \emph{恢复完全备份+差异备份。(或者换一种问法,即,恢复到刚刚插入新员工的位置)}

\qquad 这种恢复方式可以恢复到差异备份时的状态,即插入新员工之后、删除员工之前的状态。

\qquad 根据前面的恢复完整备份+差异备份模板可得到答案:

\begin{mdframed}[backgroundcolor=blue!5]
\begin{verbatim}
-- 方式1:使用备份设备恢复完整备份
RESTORE DATABASE 销售管理 FROM abc109
WITH REPLACE, NORECOVERY

RESTORE DATABASE 销售管理
FROM DISK = 'D:\abc\销售管理_差异备份.bak'
WITH RECOVERY

-- 方式2:完全使用文件路径
RESTORE DATABASE 销售管理
FROM DISK = 'D:\abc\abc109.bak'
WITH REPLACE, NORECOVERY

RESTORE DATABASE 销售管理
FROM DISK = 'D:\abc\销售管理_差异备份.bak'
WITH RECOVERY
\end{verbatim}
\end{mdframed}

\subsubsection{问题六}

\textbf{题目} \emph{恢复完全备份+差异备份+日志备份。(或者换一种问法,即,恢复到刚刚删除新员工的位置)}

\qquad 这是最完整的恢复策略,可以恢复到日志备份时的精确状态,即删除员工操作完成后的状态。

\qquad 根据前面的恢复完整备份+差异备份+日志备份模板可得到答案:

\begin{mdframed}[backgroundcolor=blue!5]
\begin{verbatim}
-- 方式1:混合使用备份设备和文件路径
RESTORE DATABASE 销售管理 FROM abc109
WITH REPLACE, NORECOVERY

RESTORE DATABASE 销售管理
FROM DISK = 'D:\abc\销售管理_差异备份.bak'
WITH NORECOVERY

RESTORE LOG 销售管理
FROM DISK = 'D:\abc\销售管理_日志备份.trn'
WITH RECOVERY

-- 方式2:完全使用文件路径
RESTORE DATABASE 销售管理
FROM DISK = 'D:\abc\abc109.bak'
WITH REPLACE, NORECOVERY

RESTORE DATABASE 销售管理
FROM DISK = 'D:\abc\销售管理_差异备份.bak'
WITH NORECOVERY

RESTORE LOG 销售管理
FROM DISK = 'D:\abc\销售管理_日志备份.trn'
WITH RECOVERY
\end{verbatim}
\end{mdframed}