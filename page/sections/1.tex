\section{第一题:数据库的创建}

\subsection{题目}

\qquad 学生选课管理数据库经过一段时间的使用后,随着数据量的不断增大,引起数据库空间不足。

\begin{enumerate}
  \item 现增加一个数据文件存储在D:\textbackslash,数据文件的逻辑名称为Stu\_Data2,物理文件名为Stu\_data2.ndf,初始大小为10MB,最大尺寸为 2GB,增长速度为10MB。
  \item 现在增加一个事务日志文件,存储在D:\textbackslash 中,日志文件的逻辑名称为Stu\_log2,物理文件名为Stu\_log2.ldf,初始大小为10MB,最大尺寸为500MB,增长速率为15\%。
\end{enumerate}

\subsection{解析}

\qquad 这里给出答案:

\begin{mdframed}[backgroundcolor=blue!5]

\begin{verbatim}
    alter database 学生选课
    add file
    (
        name = Stu_Data2,
        filename = 'D:\Stu_data2.ndf',
        size = 10mb,
        maxsize = 2gb,
        filegrowth = 10mb
    )

    alter database 学生选课
    add log file
    (
        name = Stu_log2,
        filename = 'D:\Stu_log2.ldf',
        size = 10mb,
        maxsize = 500mb,
        filegrowth = 15%
    )
\end{verbatim}

\end{mdframed}

\qquad 创建数据库模板:

\begin{mdframed}[backgroundcolor=gray!10]
\begin{verbatim}
    /*创建数据库*/
    create database [这里填数据库名称]
    on
    (
        name = [这里填文件的逻辑名称],
        filename = [这里填文件的存储位置以及文件名称],
        size = [这里填文件的初始大小],
        maxsize = [这里填限制文件的最大存储大小
                (如果要求无限大填unlimited)],
        filegrowth = [这里填文件大小的增长速度]
    )
    /*接下来为log日志文件的建立*/
    log on
    (
        /*格式与创建数据库部分一致*/
    )
\end{verbatim}
\end{mdframed}

\qquad 需要注意的是,数据库文件名的格式为\texttt{*.ndf},
而日志文件名的格式为\texttt{*.ldf},注意区别文件后缀。

\qquad 添加文件的格式与创建数据库的格式是一样的:

\begin{mdframed}[backgroundcolor=gray!10]
\begin{verbatim}
    /*为数据库添加文件*/
    alter database [数据库名称]
    add [这里填文件类型,如果要添加数据库文件就填file,
        如果要添加日志文件就填log file]
    (
        /*这部分格式与创建数据库部分一致*/
    )
\end{verbatim}
\end{mdframed}

\begin{itemize}
  \item \texttt{add file} 添加数据库文件
  \item \texttt{add log file} 添加日志文件
\end{itemize}

\subsection{知识点拓展}

\subsubsection{数据库日志的作用}

\qquad 数据库日志文件(Transaction Log)是数据库系统中至关重要的组成部分,主要作用包括:

\begin{enumerate}
  \item \textbf{事务恢复}:记录所有对数据库的修改操作,在系统崩溃时可以通过日志进行数据恢复。
  \item \textbf{数据一致性}:确保事务的ACID特性,特别是原子性和一致性。
  \item \textbf{回滚操作}:当事务需要回滚时,可以根据日志中的记录撤销已执行的操作。
  \item \textbf{备份支持}:支持增量备份和时点恢复功能。
  \item \textbf{数据库镜像}:为数据库镜像和复制提供数据同步基础。
\end{enumerate}

\qquad 日志文件通常包含以下信息:
\begin{itemize}
  \item 事务的开始和结束
  \item 数据修改前后的值
  \item 检查点信息
  \item 页面分配信息
\end{itemize}

\subsubsection{增长速度的设置}

\qquad 数据库文件的增长速度设置对数据库性能和存储管理有重要影响:

\textbf{1. 固定大小增长(如10MB)}
\begin{itemize}
  \item \textbf{优点}:增长大小可预测,便于存储空间规划
  \item \textbf{缺点}:当数据量较大时,频繁的小幅增长可能影响性能
  \item \textbf{适用场景}:数据增长相对稳定的小型数据库
\end{itemize}

\textbf{2. 百分比增长(如15\%)}
\begin{itemize}
  \item \textbf{优点}:随着文件大小增大,增长量也相应增大,减少增长频率
  \item \textbf{缺点}:增长大小不可预测,可能导致存储空间不足
  \item \textbf{适用场景}:数据增长较快的大型数据库
\end{itemize}

\textbf{3. 最佳实践建议}
\begin{itemize}
  \item 数据文件:建议使用固定大小增长,避免频繁的自动增长
  \item 日志文件:可以使用百分比增长,但要设置合理的最大大小限制
  \item 预分配空间:根据预期的数据增长提前分配足够的空间
  \item 监控增长:定期监控文件增长情况,及时调整策略
\end{itemize}