\section{第一题:数据库的创建}

\subsection{题目}

\qquad 学生选课管理数据库经过一段时间的使用后,随着数据量的不断增大,引起数据库空间不足。

\begin{enumerate}
  \item 现增加一个数据文件存储在D:\textbackslash,数据文件的逻辑名称为Stu\_Data2,物理文件名为Stu\_data2.ndf,初始大小为10MB,最大尺寸为 2GB,增长速度为10MB。
  \item 现在增加一个事务日志文件,存储在D:\textbackslash 中,日志文件的逻辑名称为Stu\_log2,物理文件名为Stu\_log2.ldf,初始大小为10MB,最大尺寸为500MB,增长速率为15\%。
\end{enumerate}

\subsection{解析}

\qquad 这里给出答案:

\begin{mdframed}[backgroundcolor=blue!5]

\begin{verbatim}
    alter database 学生选课
    add file
    (
        name = Stu_Data2,
        filename = 'D:\Stu_data2.ndf',
        size = 10mb,
        maxsize = 2gb,
        filegrowth = 15%
    )

    alter database 学生选课
    add log file
    (
        name = Stu_log2,
        filename = 'D:\Stu_log2.ldf',
        size = 10mb,
        maxsize = 500mb,
        filegrowth = 15%
    )
\end{verbatim}

\end{mdframed}

\qquad 创建数据库模板:

\begin{mdframed}[backgroundcolor=gray!10]
\begin{verbatim}
    /*创建数据库*/
    create database [这里填数据库名称]
    on
    (
        name = [这里填文件的逻辑名称],
        filename = [这里填文件的存储位置以及文件名称],
        size = [这里填文件的初始大小],
        maxsize = [这里填限制文件的最大存储大小
                (如果要求无限大填unlimited)],
        filegrowth = [这里填文件大小的增长速度]
    )
    /*接下来为log日志文件的建立*/
    log on
    (
        /*格式与创建数据库部分一致*/
    )
\end{verbatim}
\end{mdframed}

\qquad 需要注意的是,数据库文件名的格式为\texttt{*.ndf},
而日志文件名的格式为\texttt{*.ldf},注意区别文件后缀。

\qquad 添加文件的格式与创建数据库的格式是一样的:

\begin{mdframed}[backgroundcolor=gray!10]
\begin{verbatim}
    /*为数据库添加文件*/
    alter database [数据库名称]
    add [这里填文件类型,如果要添加数据库文件就填file,
        如果要添加日志文件就填log file]
    (
        /*这部分格式与创建数据库部分一致*/
    )
\end{verbatim}
\end{mdframed}

\begin{itemize}
  \item \texttt{add file} 添加数据库文件
  \item \texttt{add log file} 添加日志文件
\end{itemize}