\section{第四题:数据表的创建与修改}

\subsection{题目}

\begin{enumerate}
  \item 创建下述course表

    \begin{tabular}{|l|l|l|l|}
      \hline
      列名 & 数据类型 & 宽度 & 为空性 \\
      \hline
      cno & char & 6 & \\
      \hline
      cname & char & 20 & \\
      \hline
      credit & tinyint & & $\surd$ \\
      \hline
    \end{tabular}
  \item 在course表中添加一列semester,整数型,非空
  \item 将cno设置成主键(主键名字为pk\_seme)
  \item 在sc表中将cno设置为外键(外键名字为fk\_cno)
\end{enumerate}

\subsection{解析}

\subsubsection{问题一:创建表}

\textbf{问题} \emph{创建course表}

\qquad 分析创建表的模板:

\begin{mdframed}[backgroundcolor=gray!10]
\begin{verbatim}
CREATE TABLE 表名
(
    列名1 数据类型(长度) [约束条件],
    列名2 数据类型(长度) [约束条件],
    ...
    列名n 数据类型(长度) [约束条件],
    [表级约束]
)
\end{verbatim}
\end{mdframed}

\textbf{常用数据类型:}
\begin{itemize}
  \item \textbf{char(n)}:固定长度字符串
  \item \textbf{varchar(n)}:可变长度字符串
  \item \textbf{int}:整数型
  \item \textbf{tinyint}:小整数型(0-255)
  \item \textbf{decimal(p,s)}:定点数
  \item \textbf{datetime}:日期时间型
\end{itemize}

\textbf{约束条件:}
\begin{itemize}
  \item \textbf{NOT NULL}:非空约束
  \item \textbf{NULL}:允许为空(默认)
  \item \textbf{PRIMARY KEY}:主键约束
  \item \textbf{UNIQUE}:唯一约束
  \item \textbf{DEFAULT 值}:默认值约束
\end{itemize}

\qquad 根据题目要求,创建course表的答案如下:

\begin{mdframed}[backgroundcolor=blue!5]
\begin{verbatim}
CREATE TABLE course
(
    cno CHAR(6) NOT NULL,
    cname CHAR(20) NOT NULL,
    credit TINYINT NULL
)
\end{verbatim}
\end{mdframed}

\subsubsection{问题二:添加列}

\textbf{问题} \emph{在course表中添加一列semester,整数型,非空}

\qquad 分析添加列的模板:

\begin{mdframed}[backgroundcolor=gray!10]
\begin{verbatim}
ALTER TABLE 表名
ADD 列名 数据类型(长度) [约束条件]
\end{verbatim}
\end{mdframed}

\textbf{说明:}
\begin{itemize}
  \item \textbf{ALTER TABLE}:修改表结构的关键字
  \item \textbf{ADD}:添加列的操作
  \item 可以同时添加多列,用逗号分隔
  \item 添加的列默认为NULL,如需非空需显式指定NOT NULL
\end{itemize}

\qquad 根据题目要求,添加semester列的答案如下:

\begin{mdframed}[backgroundcolor=blue!5]
\begin{verbatim}
ALTER TABLE course
ADD semester INT NOT NULL
\end{verbatim}
\end{mdframed}

\subsubsection{问题三:设置主键}

\textbf{问题} \emph{将cno设置成主键(主键名字为pk\_seme)}

\qquad 分析添加主键约束的模板:

\begin{mdframed}[backgroundcolor=gray!10]
\begin{verbatim}
ALTER TABLE 表名
ADD CONSTRAINT 约束名 PRIMARY KEY (列名)
\end{verbatim}
\end{mdframed}

\textbf{说明:}
\begin{itemize}
  \item \textbf{ADD CONSTRAINT}:添加约束的关键字
  \item \textbf{约束名}:自定义的约束名称,便于管理
  \item \textbf{PRIMARY KEY}:主键约束类型
  \item 主键列必须为NOT NULL,且值唯一
  \item 一个表只能有一个主键
\end{itemize}

\qquad 根据题目要求,设置cno为主键的答案如下:

\begin{mdframed}[backgroundcolor=blue!5]
\begin{verbatim}
ALTER TABLE course
ADD CONSTRAINT pk_seme PRIMARY KEY (cno)
\end{verbatim}
\end{mdframed}

\textbf{拓展说明}

\begin{mdframed}[backgroundcolor=yellow!10]
  \textbf{主键的作用:}
  \begin{itemize}
    \item \textbf{唯一性标识}:确保表中每一行都有唯一的标识符,不允许重复值
    \item \textbf{非空约束}:主键列不能包含NULL值,保证数据完整性
    \item \textbf{建立索引}:系统自动为主键创建唯一聚集索引,提高查询性能
    \item \textbf{外键引用}:作为其他表外键的参照目标,建立表间关系
    \item \textbf{数据完整性}:防止插入重复或无效的数据,维护数据质量
    \item \textbf{优化存储}:聚集索引按主键顺序物理存储数据,提高I/O效率
  \end{itemize}
\end{mdframed}

\subsubsection{问题四:设置外键}

\textbf{问题} \emph{在sc表中将cno设置为外键(外键名字为fk\_cno)}

\qquad 外键可以在创建表时定义,也可以在表创建后添加。

\textbf{方式一:创建表时定义外键}

\begin{mdframed}[backgroundcolor=gray!10]
\begin{verbatim}
CREATE TABLE 表名
(
    列名1 数据类型 约束条件,
    列名2 数据类型 约束条件,
    ...
    CONSTRAINT 外键名 FOREIGN KEY (列名)
        REFERENCES 参照表名(参照列名)
)
\end{verbatim}
\end{mdframed}

\textbf{方式二:表创建后添加外键}

\begin{mdframed}[backgroundcolor=gray!10]
\begin{verbatim}
ALTER TABLE 表名
ADD CONSTRAINT 外键名
    FOREIGN KEY (列名) REFERENCES 参照表名(参照列名)
\end{verbatim}
\end{mdframed}

\textbf{说明:}
\begin{itemize}
  \item \textbf{FOREIGN KEY}:定义外键约束
  \item \textbf{REFERENCES}:指定参照的主表和主键
  \item 外键列的数据类型必须与参照列相同
  \item 参照列必须是主键或唯一键
\end{itemize}

\qquad 根据题目要求,在sc表中设置cno为外键的答案如下:

\begin{mdframed}[backgroundcolor=blue!5]
\begin{verbatim}
ALTER TABLE sc
ADD CONSTRAINT fk_cno
    FOREIGN KEY (cno) REFERENCES course(cno)
\end{verbatim}
\end{mdframed}

\begin{mdframed}[backgroundcolor=yellow!10]
  \textbf{外键的作用:}
  \begin{itemize}
    \item \textbf{维护参照完整性}:确保子表中的外键值必须在父表的主键中存在
    \item \textbf{防止数据不一致}:避免插入无效的关联数据
    \item \textbf{级联操作}:可设置级联删除或更新,保持数据同步
    \item \textbf{建立表间关系}:明确表与表之间的逻辑关系
    \item \textbf{约束数据操作}:限制对父表主键的删除和修改操作
    \item \textbf{提高数据可靠性}:防止孤立记录的产生
  \end{itemize}
\end{mdframed}

\subsection{知识点拓展}

\subsubsection{唯一约束}

\qquad 唯一约束确保列中的值是唯一的,但允许NULL值。

\begin{mdframed}[backgroundcolor=gray!10]
\begin{verbatim}
-- 创建表时定义
CREATE TABLE 表名
(
    列名 数据类型 UNIQUE,
    ...
)

-- 后续添加
ALTER TABLE 表名
ADD CONSTRAINT 约束名 UNIQUE (列名)
\end{verbatim}
\end{mdframed}

\textbf{示例:}
\begin{verbatim}
ALTER TABLE student
ADD CONSTRAINT uk_student_email UNIQUE (email);
\end{verbatim}

\subsubsection{检查约束}

\qquad 检查约束用于限制列中允许的值范围。

\begin{mdframed}[backgroundcolor=gray!10]
\begin{verbatim}
-- 创建表时定义
CREATE TABLE 表名
(
    列名 数据类型 CHECK (条件表达式),
    ...
)

-- 后续添加
ALTER TABLE 表名
ADD CONSTRAINT 约束名 CHECK (条件表达式)
\end{verbatim}
\end{mdframed}

\textbf{示例:}
\begin{verbatim}
ALTER TABLE student
ADD CONSTRAINT ck_student_age CHECK (age >= 0 AND age <= 150);
\end{verbatim}

\subsubsection{默认值约束}

\qquad 默认值约束为列提供默认值,在插入数据时如果未指定值则使用默认值。

\begin{mdframed}[backgroundcolor=gray!10]
\begin{verbatim}
-- 创建表时定义
CREATE TABLE 表名
(
    列名 数据类型 DEFAULT 默认值,
    ...
)

-- 后续添加
ALTER TABLE 表名
ADD CONSTRAINT 约束名 DEFAULT 默认值 FOR 列名
\end{verbatim}
\end{mdframed}

\textbf{示例:}
\begin{verbatim}
ALTER TABLE student
ADD CONSTRAINT df_student_status DEFAULT '在读' FOR status;
\end{verbatim}

\subsubsection{约束禁用和启用}

\qquad 可以临时禁用或启用约束,常用于数据导入或维护操作。

\begin{mdframed}[backgroundcolor=gray!10]
\begin{verbatim}
-- 禁用约束
ALTER TABLE 表名 NOCHECK CONSTRAINT 约束名;

-- 启用约束
ALTER TABLE 表名 CHECK CONSTRAINT 约束名;

-- 删除约束
ALTER TABLE 表名 DROP CONSTRAINT 约束名;
\end{verbatim}
\end{mdframed}

\textbf{示例:}
\begin{verbatim}
-- 禁用外键约束
ALTER TABLE sc NOCHECK CONSTRAINT fk_cno;

-- 启用外键约束
ALTER TABLE sc CHECK CONSTRAINT fk_cno;

-- 删除约束
ALTER TABLE sc DROP CONSTRAINT fk_cno;
\end{verbatim}

\begin{mdframed}[backgroundcolor=yellow!10]
  \textbf{约束管理要点:}
  \begin{itemize}
    \item 约束名称应具有描述性,便于识别和管理
    \item 禁用约束后记得及时启用,避免数据完整性问题
    \item 删除约束前要谨慎考虑,确保不会影响数据完整性
    \item 可以通过系统视图查询表中的所有约束信息
  \end{itemize}
\end{mdframed}