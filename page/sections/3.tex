\section{第三题:触发器}

\subsection{题目}

\begin{enumerate}
  \item 通过创建一个后触发型触发器tr\_sc\_del,限制删除‘计算机系’学生的信息,并给出信息“不能删除计算机系学生选课信息!”(通过多表连接,不要用别名)
  \item 通过创建一个前触发型触发器tr\_stu\_del,限制删除有选课的学生信息。(注意:通过内连接实现,不取别名,临时表在前)
  \item 创建一个用于防止用户删除学生选课数据库中任何数据表的触发器tr\_droptable。
  \item 为了必须删除一个选课记录(学号95001, 课程号001),请先抑制触发器tr\_sc\_del,删除后,再恢复触发器。
\end{enumerate}

\subsection{解析}

\subsubsection{问题一}

\textbf{题目} \emph{通过创建一个后触发型触发器tr\_sc\_del,限制删除‘计算机系’学生的信息,并给出信息“不能删除计算机系学生选课信息!”(通过多表连接,不要用别名)}

\vspace{6pt}

\qquad 先来介绍一下触发器。

\textbf{1. DML触发器}

\qquad DML触发器分为三类:AFTER触发器、INSTEAD OF触发器和CLR触发器。

\begin{itemize}
  \item \textbf{AFTER触发器}:在触发事件(INSERT、UPDATE、DELETE)完成后执行,是最常用的触发器类型
  \item \textbf{INSTEAD OF触发器}:替代触发事件执行,主要用于视图上的数据修改操作
  \item \textbf{CLR触发器}:使用.NET Framework公共语言运行时创建的触发器,允许使用托管代码编写
\end{itemize}

\textbf{2. DDL触发器}

\qquad DDL触发器是在数据库结构发生变化时触发的,如创建、修改或删除表、索引、视图等操作。与DML触发器不同,DDL触发器主要用于数据库管理和安全控制。

\begin{itemize}
  \item \textbf{触发事件}:CREATE、ALTER、DROP等数据定义语言操作
  \item \textbf{作用范围}:可以在数据库级别或服务器级别设置
  \item \textbf{主要用途}:防止误操作、记录结构变更日志、实施安全策略
\end{itemize}

\vspace{6pt}

\qquad 来重点讲一下DML触发器

\textbf{1. INSERTED表和DELETED表}

\qquad 在DML触发器中,系统会自动创建两张临时表:INSERTED表和DELETED表,用于存储触发器执行前后的数据变化。

\begin{itemize}
  \item \textbf{INSERTED表}:存储新插入或更新后的数据
  \item \textbf{DELETED表}:存储被删除或更新前的数据
\end{itemize}

\textbf{不同操作中的表状态:}

\begin{enumerate}
  \item \textbf{INSERT操作}:
    \begin{itemize}
      \item INSERTED表:包含新插入的行数据
      \item DELETED表:为空
    \end{itemize}

  \item \textbf{DELETE操作}:
    \begin{itemize}
      \item INSERTED表:为空
      \item DELETED表:包含被删除的行数据
    \end{itemize}

  \item \textbf{UPDATE操作}:
    \begin{itemize}
      \item INSERTED表:包含更新后的新数据
      \item DELETED表:包含更新前的旧数据
    \end{itemize}
\end{enumerate}

\textbf{总结过程:} \textcolor{red}{表中被删除的数据会被转移到DELETED表,
插入表中的新数据会被转移到INSERTED表。}

\begin{mdframed}[backgroundcolor=gray!10]
  \textbf{使用示例:}
\begin{verbatim}
-- 在触发器中访问这两张表
SELECT * FROM INSERTED;  -- 查看新数据
SELECT * FROM DELETED;   -- 查看旧数据

-- 常用于数据验证和日志记录
IF EXISTS (SELECT * FROM INSERTED WHERE salary < 0)
BEGIN
    RAISERROR('工资不能为负数', 16, 1)
    ROLLBACK TRANSACTION
END
\end{verbatim}
\end{mdframed}

\textbf{2. 创建触发器}

\qquad 分析创建DML触发器的模板:

\begin{mdframed}[backgroundcolor=gray!10]
\begin{verbatim}
create trigger 触发器名
on 表名或视图名
{for | after | instead of} -- 这里是选择触发器的类型
                           -- 如果仅指定 for 关键字,则after为默认值
{insert | update | delete} -- 这里是指定哪种数据操作将激活触发器
[with encryption] -- 如果要求加密触发器需要加上这句
as
[if update (列名)] -- 判断指定的列(列名)是否进行了插入或更新操作
sql_statements -- 其他需要执行的sql语句
-- 注:单条语句时可省略BEGIN...END,多条语句时需要使用
\end{verbatim}
\end{mdframed}

\textbf{3. 触发器执行顺序}

\qquad 当数据库操作发生时,触发器的执行遵循以下顺序:

\begin{enumerate}
  \item \textbf{执行约束检查}:检查主键、外键、唯一约束等
  \item \textbf{执行INSTEAD OF触发器}:如果存在,替代原始操作执行
  \item \textbf{执行数据操作}:如果没有INSTEAD OF触发器,执行原始的INSERT/UPDATE/DELETE操作
  \item \textbf{执行AFTER触发器}:在数据操作完成后执行
  \item \textbf{提交或回滚事务}:根据触发器执行结果决定
\end{enumerate}

\begin{mdframed}[backgroundcolor=yellow!10]
  \textbf{重要提示:}
  \begin{itemize}
    \item 如果触发器中执行了\texttt{ROLLBACK TRANSACTION},整个事务(包括原始操作)都会被回滚
    \item 多个触发器存在时,执行顺序可以通过\texttt{sp\_settriggerorder}存储过程来设置
    \item 触发器中的错误会导致\textcolor{red}{整个事务失败}
    \item AFTER触发器也叫做\textcolor{red}{后触发型触发器}
    \item INSTEAD OF触发器也叫做\textcolor{red}{前触发型触发器}
  \end{itemize}
\end{mdframed}

\vspace{6pt}

\qquad 现在可以得到这题的答案:

\begin{mdframed}[backgroundcolor=blue!5]
\begin{verbatim}
create trigger tr_sc_del
on SC
after update
as
if exists(
    select 1
    from delete
    inner join stu on delete.sno = stu.sno
    where stu.sdept = '计算机系'
    -- 查看被删除的信息中有没有计算机系的学生的信息
)
begin
    print '不能删除计算机系学生选课信息!'; -- 打印错误信息
    rollback transaction; -- 回滚事务
    return; -- 退出触发器
end
\end{verbatim}
\end{mdframed}

\subsubsection{问题二}

\textbf{题目} \emph{通过创建一个前触发型触发器tr\_stu\_del,限制删除有选课的学生信息。(注意:通过内连接实现,不取别名,临时表在前)}

\vspace{6pt}

\qquad 分析题目可以知道要使用前触发型触发器,在表被修改前进行操作。
根据上一题的解析可以得到答案:

\begin{mdframed}[backgroundcolor=blue!5]
\begin{verbatim}
create trigger tr_stu_del
on stu
instead of delete
as
if exists(
    select 1
    from delete
    inner join sc on delete.sno = sc.sno
    -- 查找要删除的信息中是否存在有选课记录的学生
)
begin
    print '不能删除有选课记录的学生信息!';
    return;
end
else
-- INSTEAD型触发器会拦截删除操作,所以需要手动恢复
begin
    delete from stu
    where sno in (select sno from deleted);
end
\end{verbatim}
\end{mdframed}

\vspace{6pt}

\begin{mdframed}[backgroundcolor=yellow!10]
  \textbf{重要提示:如何判断在哪张表上创建触发器}

  \begin{itemize}
    \item \textbf{看操作对象}:题目中提到要限制对哪张表的操作,触发器就创建在那张表上
    \item \textbf{问题一}:限制删除"计算机系学生的选课信息" → 操作的是SC表 → 在SC表上创建触发器
    \item \textbf{问题二}:限制删除"有选课的学生信息" → 操作的是Student表 → 在Student表上创建触发器
    \item \textbf{一般规律}:
      \begin{itemize}
        \item 如果要控制对表A的INSERT/UPDATE/DELETE操作,就在表A上创建触发器
        \item 触发器会在对该表进行指定操作时自动执行
        \item 通过INSERTED/DELETED表可以获取操作前后的数据进行判断
      \end{itemize}
  \end{itemize}
\end{mdframed}

\subsubsection{问题三}

\textbf{题目} \emph{创建一个用于防止用户删除学生选课数据库中任何数据表的触发器tr\_droptable}

\vspace{6pt}

\qquad 根据题目可以发现,触发器触发条件是删除数据表,由此可以判断需要创建DDL触发器。

\qquad 分析DDL触发器的创建模板:

\begin{mdframed}[backgroundcolor=gray!10]
\begin{verbatim}
create trigger 触发器名
on {all server | database}
[with encryption]
{for | after} {DDL事件} [,...n]
as
    sql_statements
\end{verbatim}
\end{mdframed}

\qquad 每一个DDL事件都对应一个Transact-SQL语句。例如,DROP\_TABLE事件对应DROP TABLE语句,CREATE\_TABLE事件对应CREATE TABLE语句,ALTER\_TABLE事件对应ALTER TABLE语句等。

\qquad 根据题目要求,我们需要防止删除数据表,所以使用DROP\_TABLE事件。答案如下:

\begin{mdframed}[backgroundcolor=blue!5]
\begin{verbatim}
create trigger tr_droptable
on database
for drop_table
as
begin
    print '禁止删除数据库中的表!';
    rollback transaction;
end
\end{verbatim}
\end{mdframed}

\subsubsection{问题四}

\textbf{题目} \emph{为了必须删除一个选课记录(学号95001, 课程号001),请先抑制触发器tr\_sc\_del,删除后,再恢复触发器。}

\vspace{6pt}

\qquad 分析题目可知,要删除一个值需要先禁用触发器,删除后再启用这个触发器,
考察触发器的禁用和启用。

\vspace{6pt}

\textbf{1. 禁用触发器}

\qquad 禁用触发器的模板:

\begin{mdframed}[backgroundcolor=gray!10]
\begin{verbatim}
disable trigger {all | 触发器名 [,...n]} -- 哪个触发器
on {object_name | database | all server} -- 在哪个位置的触发器
\end{verbatim}
\end{mdframed}

\textbf{2. 启用触发器}

\qquad 启用触发器的模板:

\begin{mdframed}[backgroundcolor=gray!10]
\begin{verbatim}
enable trigger {all | 触发器名 [,...n]} -- 哪个触发器
on {object_name | database | all server} -- 在哪个位置的触发器
\end{verbatim}
\end{mdframed}

\qquad 根据题目要求,完整的操作步骤如下:

\begin{mdframed}[backgroundcolor=blue!5]
\begin{verbatim}
-- 第一步:禁用触发器
disable trigger tr_sc_del on sc;

-- 第二步:删除指定的选课记录
delete from sc
where sno = '95001' and cno = '001';

-- 第三步:启用触发器
enable trigger tr_sc_del on sc;
\end{verbatim}
\end{mdframed}

\subsection{知识点拓展}

\subsubsection{查看触发器}

\qquad 可以通过以下几种方式查看触发器:

\begin{enumerate}
  \item \textbf{查看数据库中的所有触发器}:
    \begin{verbatim}
    SELECT * FROM sys.triggers;
    \end{verbatim}

  \item \textbf{查看特定表上的触发器}:
    \begin{verbatim}
    SELECT * FROM sys.triggers
    WHERE parent_id = OBJECT_ID('表名');
    \end{verbatim}

  \item \textbf{查看触发器的详细信息}:
    \begin{verbatim}
    EXEC sp_helptext '触发器名';
    \end{verbatim}
\end{enumerate}

\subsubsection{删除触发器}

\qquad 删除触发器的语法很简单:

\begin{mdframed}[backgroundcolor=gray!10]
\begin{verbatim}
DROP TRIGGER 触发器名;
\end{verbatim}
\end{mdframed}

\qquad 示例:
\begin{verbatim}
-- 删除DML触发器
DROP TRIGGER tr_sc_del;

-- 删除DDL触发器
DROP TRIGGER tr_droptable ON DATABASE;
\end{verbatim}

\textbf{注意:}删除触发器后,该触发器的所有功能将永久失效,请谨慎操作。
