\documentclass[11pt,a4paper]{article}

\usepackage{ctex}
\usepackage{float}
\usepackage[utf8]{inputenc}
\usepackage{graphicx}
\usepackage{subcaption}
\usepackage{multirow}
\usepackage{enumitem}
\usepackage{tabularx}
\usepackage{booktabs}
\usepackage{geometry}
\geometry{margin=2.5cm}

\usepackage{listings}
\usepackage[dvipsnames]{xcolor}
\usepackage{fontspec} % 用于设置字体
\setmonofont{JetBrains Mono}[Scale=0.85] % 设置等宽字体为JetBrains Mono
\usepackage[framemethod=TikZ]{mdframed}
\usepackage{tikz}  % 添加TikZ包用于绘图
\usetikzlibrary{shapes,positioning}
\usepackage{hyperref}  % 添加超链接支持,使目录可点击跳转
\hypersetup{
  colorlinks=true,  % 启用彩色链接
  linkcolor=blue,   % 内部链接颜色(目录、交叉引用等)
  urlcolor=blue,    % URL链接颜色
  citecolor=blue,   % 引用链接颜色
  bookmarksdepth=3, % PDF书签深度
  pdfstartview=FitH % PDF打开时的视图方式
}
% 在导言区进行样式设置
\lstset{
  language=SQL, % 设置语言
  basicstyle=\ttfamily\small, % 设置字体族和大小
  breaklines=true, % 自动换行
  keywordstyle=\bfseries\color{Purple}, % C语言关键字:紫色粗体
  morekeywords={uint32_t,uint8_t,float,char,static,const}, % C语言扩展关键字
  emph=[1]{delay_init,key_init,lcd_init,infrared_init,rgb_init,buzzer_init,led_init,usart_init,htu21d_init,fan_pwm_init}, % 初始化函数
  emphstyle=[1]\bfseries\color{Red}, % 初始化函数:红色粗体
  emph=[2]{htu21d_t,htu21d_h,get_infrared_status,fan_pwm_control,rgb_ctrl,led_control,buzzer_tweet}, % 硬件控制函数
  emphstyle=[2]\bfseries\color{Orange}, % 硬件控制函数:橙色粗体
  emph=[3]{update_display,get_centered_x,check_uart_command,send_system_status,auto_control,check_human_presence}, % 逻辑函数
  emphstyle=[3]\bfseries\color{Blue}, % 逻辑函数:蓝色粗体
  emph=[4]{sprintf,printf,strncmp,strlen,delay_ms,delay_count}, % 标准库函数
  emphstyle=[4]\color{Magenta}, % 标准库函数:洋红色
  commentstyle=\itshape\color{ForestGreen}, % 注释:森林绿斜体
  stringstyle=\color{Maroon}, % 字符串:栗色
  identifierstyle=\color{Black}, % 标识符:黑色
  numberstyle=\tiny\color{Gray}, % 数字:灰色小字
  columns=flexible,
  numbers=none, % 不显示行号
  % numbersep=1.5em, % 设置行号的具体位置
  % numberstyle=\footnotesize\color{Gray}, % 行号样式
  frame=single, % 边框
  framesep=1em, % 设置代码与边框的距离
  rulecolor=\color{lightgray}, % 边框颜色
  %backgroundcolor=\color{Gray!10}, % 背景色:淡灰色
  tabsize=4, % Tab键宽度
  showstringspaces=false % 不显示字符串中的空格标记
}

\begin{document}
{
  \centering
  \fontsize{20pt}{24pt}\selectfont
  \begin{minipage}[c]{0.7\textwidth}
    \centering
    数据库期末复习笔记
  \end{minipage}%
  \hfill
  \begin{minipage}[c]{0.2\textwidth}
    \raggedleft
    \includegraphics[width=3cm]{sticker.pdf}
  \end{minipage}
}

\tableofcontents

\raggedright

\newpage

\section{第一题:数据库的创建}

\subsection{题目}

\qquad 学生选课管理数据库经过一段时间的使用后,随着数据量的不断增大,引起数据库空间不足。

\begin{enumerate}
  \item 现增加一个数据文件存储在D:\textbackslash,数据文件的逻辑名称为Stu\_Data2,物理文件名为Stu\_data2.ndf,初始大小为10MB,最大尺寸为 2GB,增长速度为10MB。
  \item 现在增加一个事务日志文件,存储在D:\textbackslash 中,日志文件的逻辑名称为Stu\_log2,物理文件名为Stu\_log2.ldf,初始大小为10MB,最大尺寸为500MB,增长速率为15\%。
\end{enumerate}

\subsection{解析}

\qquad 这里给出答案:

\begin{mdframed}[backgroundcolor=blue!5]

\begin{verbatim}
    alter database 学生选课
    add file
    (
        name = Stu_Data2,
        filename = 'D:\Stu_data2.ndf',
        size = 10mb,
        maxsize = 2gb,
        filegrowth = 10mb
    )

    alter database 学生选课
    add log file
    (
        name = Stu_log2,
        filename = 'D:\Stu_log2.ldf',
        size = 10mb,
        maxsize = 500mb,
        filegrowth = 15%
    )
\end{verbatim}

\end{mdframed}

\qquad 创建数据库模板:

\begin{mdframed}[backgroundcolor=gray!10]
\begin{verbatim}
    /*创建数据库*/
    create database [这里填数据库名称]
    on
    (
        name = [这里填文件的逻辑名称],
        filename = [这里填文件的存储位置以及文件名称],
        size = [这里填文件的初始大小],
        maxsize = [这里填限制文件的最大存储大小
                (如果要求无限大填unlimited)],
        filegrowth = [这里填文件大小的增长速度]
    )
    /*接下来为log日志文件的建立*/
    log on
    (
        /*格式与创建数据库部分一致*/
    )
\end{verbatim}
\end{mdframed}

\qquad 需要注意的是,数据库文件名的格式为\texttt{*.ndf},
而日志文件名的格式为\texttt{*.ldf},注意区别文件后缀。

\qquad 添加文件的格式与创建数据库的格式是一样的:

\begin{mdframed}[backgroundcolor=gray!10]
\begin{verbatim}
    /*为数据库添加文件*/
    alter database [数据库名称]
    add [这里填文件类型,如果要添加数据库文件就填file,
        如果要添加日志文件就填log file]
    (
        /*这部分格式与创建数据库部分一致*/
    )
\end{verbatim}
\end{mdframed}

\begin{itemize}
  \item \texttt{add file} 添加数据库文件
  \item \texttt{add log file} 添加日志文件
\end{itemize}

\subsection{知识点拓展}

\subsubsection{数据库日志的作用}

\qquad 数据库日志文件(Transaction Log)是数据库系统中至关重要的组成部分,主要作用包括:

\begin{enumerate}
  \item \textbf{事务恢复}:记录所有对数据库的修改操作,在系统崩溃时可以通过日志进行数据恢复。
  \item \textbf{数据一致性}:确保事务的ACID特性,特别是原子性和一致性。
  \item \textbf{回滚操作}:当事务需要回滚时,可以根据日志中的记录撤销已执行的操作。
  \item \textbf{备份支持}:支持增量备份和时点恢复功能。
  \item \textbf{数据库镜像}:为数据库镜像和复制提供数据同步基础。
\end{enumerate}

\qquad 日志文件通常包含以下信息:
\begin{itemize}
  \item 事务的开始和结束
  \item 数据修改前后的值
  \item 检查点信息
  \item 页面分配信息
\end{itemize}

\subsubsection{增长速度的设置}

\qquad 数据库文件的增长速度设置对数据库性能和存储管理有重要影响:

\textbf{1. 固定大小增长(如10MB)}
\begin{itemize}
  \item \textbf{优点}:增长大小可预测,便于存储空间规划
  \item \textbf{缺点}:当数据量较大时,频繁的小幅增长可能影响性能
  \item \textbf{适用场景}:数据增长相对稳定的小型数据库
\end{itemize}

\textbf{2. 百分比增长(如15\%)}
\begin{itemize}
  \item \textbf{优点}:随着文件大小增大,增长量也相应增大,减少增长频率
  \item \textbf{缺点}:增长大小不可预测,可能导致存储空间不足
  \item \textbf{适用场景}:数据增长较快的大型数据库
\end{itemize}

\textbf{3. 最佳实践建议}
\begin{itemize}
  \item 数据文件:建议使用固定大小增长,避免频繁的自动增长
  \item 日志文件:可以使用百分比增长,但要设置合理的最大大小限制
  \item 预分配空间:根据预期的数据增长提前分配足够的空间
  \item 监控增长:定期监控文件增长情况,及时调整策略
\end{itemize}

\section{第二题:视图与索引}

\subsection{题目}

\qquad 视图与索引

\begin{enumerate}
  \item 创建带加密选课视图view\_sc,列名依次显示为 学号,姓名,年龄
  \item 在学生表中的sname上创建非聚集索引(Sname\_ind)
  \item 在给课程名创建唯一非聚集索引(Cname\_ind)
\end{enumerate}

\subsection{解析}
\qquad 三个问题,分步进行解析。

\subsubsection{第一问}

\textbf{题目} \emph{创建带加密选课视图view\_sc,列名依次显示为 学号,姓名,年龄。}

\vspace{6pt}

\qquad 分析一下创建视图的模板:

\begin{mdframed}[backgroundcolor=gray!10]
\begin{verbatim}
create view [视图名称]([列名显示:name1,name2,...])
/*列名显示部分设置的列名一一对应select选出来的列*/
[with encryption] /*如果要求创建带加密的视图,需要添加这一句*/
as
/*下面为需要展示的内容的select语句*/
select [col1,col2,...]
...
\end{verbatim}
\end{mdframed}

\qquad 根据创建视图的模板可以得到答案:

\begin{mdframed}[backgroundcolor=blue!5]
\begin{verbatim}
create view view_sc(学号,姓名,年龄)
with encryption
as
select sno,sname,sage
from student
\end{verbatim}
\end{mdframed}

\subsubsection{第二问}

\textbf{题目} \emph{在学生表中的sname上创建非聚集索引(Sname\_ind)}

\qquad 分析一下创建索引的模板:

\begin{mdframed}[backgroundcolor=gray!10]
  \label{index:1}
\begin{verbatim}
create [unique 唯一索引] [clustered 聚集索引| nonclustered 非聚集索引]
/*如果不填默认为非聚集索引 nonclustered*/
index [填索引名称]
on [表名|视图名](这个表或视图中的列名:[col1,col2,...])
\end{verbatim}
\end{mdframed}

\qquad 根据模板可以得到答案:

\begin{mdframed}[backgroundcolor=blue!5]
\begin{verbatim}
create index Sname_ind
on Student(sname)
\end{verbatim}
\end{mdframed}

\subsubsection{第三问}

\textbf{题目} \emph{在给课程名创建唯一非聚集索引(Cname\_ind)}

\qquad 根据创建索引的模板\ref{index:1}可以得到答案:

\begin{mdframed}[backgroundcolor=blue!5]
\begin{verbatim}
create unique nonclustered /*题目要求唯一,需要加unique*/
index Cname_ind
on Course(Cname)
\end{verbatim}
\end{mdframed}

\subsection{知识点拓展}

\subsubsection{索引类型及特征}

\paragraph{聚集索引(Clustered Index)}
\begin{itemize}
  \item \textbf{特征}:数据行的物理存储顺序与索引键值的逻辑顺序相同
  \item \textbf{限制}:每个表只能有一个聚集索引
  \item \textbf{作用}:提高范围查询和排序操作的效率
  \item \textbf{适用场景}:经常进行范围查询的列,如主键、日期列
\end{itemize}

\paragraph{非聚集索引(Non-clustered Index)}
\begin{itemize}
  \item \textbf{特征}:索引的逻辑顺序与数据行的物理存储顺序无关
  \item \textbf{限制}:每个表可以有多个非聚集索引(最多999个)
  \item \textbf{作用}:提高特定列的查询速度,但不改变数据的物理存储
  \item \textbf{适用场景}:经常用于WHERE子句、JOIN条件的列
\end{itemize}

\paragraph{唯一索引(Unique Index)}
\begin{itemize}
  \item \textbf{特征}:确保索引键值的唯一性,不允许重复值
  \item \textbf{作用}:既提高查询性能,又保证数据完整性
  \item \textbf{适用场景}:需要保证唯一性的列,如身份证号、学号等
\end{itemize}

\paragraph{复合索引(Composite Index)}
\begin{itemize}
  \item \textbf{特征}:基于多个列创建的索引
  \item \textbf{作用}:提高多列组合查询的效率
  \item \textbf{注意}:遵循"最左前缀"原则,索引列的顺序很重要
\end{itemize}

\subsubsection{索引使用原则}

\begin{enumerate}
  \item \textbf{选择性原则}:在选择性高(重复值少)的列上创建索引效果更好
  \item \textbf{频率原则}:在经常用于查询条件的列上创建索引
  \item \textbf{维护成本}:索引会增加INSERT、UPDATE、DELETE操作的开销
  \item \textbf{存储空间}:索引需要额外的存储空间
\end{enumerate}

\begin{mdframed}[backgroundcolor=yellow!10]
  \textbf{小贴士:}索引是一把双刃剑,能够显著提高查询性能,但也会增加数据修改的开销和存储空间的占用。因此需要根据实际的查询需求和数据更新频率来合理设计索引策略。
\end{mdframed}

\subsubsection{索引的管理方式}

\textbf{1. 删除索引}

\begin{mdframed}[backgroundcolor=gray!10]
\begin{verbatim}
drop index [表名.索引名 | 视图名.索引名]

/*一个简单的例子:删除第二问中的Sname_ind索引*/
drop index Student.Sname_ind
\end{verbatim}
\end{mdframed}

\textbf{注意:}\texttt{表在创建时如果对某一列进行了主键约束会自动自动创建一个索引,这个索引是无法用drop index进行删除的。}

\vspace{6pt}

\textbf{2. 查看索引}

\qquad 可以使用SSMS软件进行查看,也可以用代码进行查看。这里介绍一下代码怎么查看。

\begin{mdframed}[backgroundcolor=gray!10]
\begin{verbatim}
use [数据库名]
go
sp_helpindex [表名]
go
\end{verbatim}
\end{mdframed}

\subsubsection{重命名索引}

\qquad 利用sp\_rename可以重命名索引

\begin{mdframed}[backgroundcolor=gray!10]
\begin{verbatim}
use [数据库名]
go
sp_rename '表名.原索引名', '新索引名'
go

/*举一个例子*/
use 学生选课
go
sp_rename 'student.Sname_index', 'index_Sname'
go
\end{verbatim}
\end{mdframed}

\section{第三题:触发器}

\subsection{题目}

\begin{enumerate}
  \item 通过创建一个后触发型触发器tr\_sc\_del,限制删除‘计算机系’学生的信息,并给出信息“不能删除计算机系学生选课信息!”(通过多表连接,不要用别名)
  \item 通过创建一个前触发型触发器tr\_stu\_del,限制删除有选课的学生信息。(注意:通过内连接实现,不取别名,临时表在前)
  \item 创建一个用于防止用户删除学生选课数据库中任何数据表的触发器tr\_droptable。
  \item 为了必须删除一个选课记录(学号95001, 课程号001),请先抑制触发器tr\_sc\_del,删除后,再恢复触发器。
\end{enumerate}

\subsection{解析}

\subsubsection{问题一}

\textbf{题目} \emph{通过创建一个后触发型触发器tr\_sc\_del,限制删除‘计算机系’学生的信息,并给出信息“不能删除计算机系学生选课信息!”(通过多表连接,不要用别名)}

\vspace{6pt}

\qquad 先来介绍一下触发器。

\textbf{1. DML触发器}

\qquad DML触发器分为三类:AFTER触发器、INSTEAD OF触发器和CLR触发器。

\begin{itemize}
  \item \textbf{AFTER触发器}:在触发事件(INSERT、UPDATE、DELETE)完成后执行,是最常用的触发器类型
  \item \textbf{INSTEAD OF触发器}:替代触发事件执行,主要用于视图上的数据修改操作
  \item \textbf{CLR触发器}:使用.NET Framework公共语言运行时创建的触发器,允许使用托管代码编写
\end{itemize}

\textbf{2. DDL触发器}

\qquad DDL触发器是在数据库结构发生变化时触发的,如创建、修改或删除表、索引、视图等操作。与DML触发器不同,DDL触发器主要用于数据库管理和安全控制。

\begin{itemize}
  \item \textbf{触发事件}:CREATE、ALTER、DROP等数据定义语言操作
  \item \textbf{作用范围}:可以在数据库级别或服务器级别设置
  \item \textbf{主要用途}:防止误操作、记录结构变更日志、实施安全策略
\end{itemize}

\vspace{6pt}

\qquad 来重点讲一下DML触发器

\textbf{1. INSERTED表和DELETED表}

\qquad 在DML触发器中,系统会自动创建两张临时表:INSERTED表和DELETED表,用于存储触发器执行前后的数据变化。

\begin{itemize}
  \item \textbf{INSERTED表}:存储新插入或更新后的数据
  \item \textbf{DELETED表}:存储被删除或更新前的数据
\end{itemize}

\textbf{不同操作中的表状态:}

\begin{enumerate}
  \item \textbf{INSERT操作}:
    \begin{itemize}
      \item INSERTED表:包含新插入的行数据
      \item DELETED表:为空
    \end{itemize}

  \item \textbf{DELETE操作}:
    \begin{itemize}
      \item INSERTED表:为空
      \item DELETED表:包含被删除的行数据
    \end{itemize}

  \item \textbf{UPDATE操作}:
    \begin{itemize}
      \item INSERTED表:包含更新后的新数据
      \item DELETED表:包含更新前的旧数据
    \end{itemize}
\end{enumerate}

\textbf{总结过程:} \textcolor{red}{表中被删除的数据会被转移到DELETED表,
插入表中的新数据会被转移到INSERTED表。}

\begin{mdframed}[backgroundcolor=gray!10]
  \textbf{使用示例:}
\begin{verbatim}
-- 在触发器中访问这两张表
SELECT * FROM INSERTED;  -- 查看新数据
SELECT * FROM DELETED;   -- 查看旧数据

-- 常用于数据验证和日志记录
IF EXISTS (SELECT * FROM INSERTED WHERE salary < 0)
BEGIN
    RAISERROR('工资不能为负数', 16, 1)
    ROLLBACK TRANSACTION
END
\end{verbatim}
\end{mdframed}

\textbf{2. 创建触发器}

\qquad 分析创建DML触发器的模板:

\begin{mdframed}[backgroundcolor=gray!10]
\begin{verbatim}
create trigger 触发器名
on 表名或视图名
{for | after | instead of} -- 这里是选择触发器的类型
                           -- 如果仅指定 for 关键字,则after为默认值
{insert | update | delete} -- 这里是指定哪种数据操作将激活触发器
[with encryption] -- 如果要求加密触发器需要加上这句
as
[if update (列名)] -- 判断指定的列(列名)是否进行了插入或更新操作
sql_statements -- 其他需要执行的sql语句
-- 注:单条语句时可省略BEGIN...END,多条语句时需要使用
\end{verbatim}
\end{mdframed}

\textbf{3. 触发器执行顺序}

\qquad 当数据库操作发生时,触发器的执行遵循以下顺序:

\begin{enumerate}
  \item \textbf{执行约束检查}:检查主键、外键、唯一约束等
  \item \textbf{执行INSTEAD OF触发器}:如果存在,替代原始操作执行
  \item \textbf{执行数据操作}:如果没有INSTEAD OF触发器,执行原始的INSERT/UPDATE/DELETE操作
  \item \textbf{执行AFTER触发器}:在数据操作完成后执行
  \item \textbf{提交或回滚事务}:根据触发器执行结果决定
\end{enumerate}

\begin{mdframed}[backgroundcolor=yellow!10]
  \textbf{重要提示:}
  \begin{itemize}
    \item 如果触发器中执行了\texttt{ROLLBACK TRANSACTION},整个事务(包括原始操作)都会被回滚
    \item 多个触发器存在时,执行顺序可以通过\texttt{sp\_settriggerorder}存储过程来设置
    \item 触发器中的错误会导致\textcolor{red}{整个事务失败}
    \item AFTER触发器也叫做\textcolor{red}{后触发型触发器}
    \item INSTEAD OF触发器也叫做\textcolor{red}{前触发型触发器}
  \end{itemize}
\end{mdframed}

\vspace{6pt}

\qquad 现在可以得到这题的答案:

\begin{mdframed}[backgroundcolor=blue!5]
\begin{verbatim}
create trigger tr_sc_del
on SC
after update
as
if exists(
    select 1
    from delete
    inner join stu on delete.sno = stu.sno
    where stu.sdept = '计算机系'
    -- 查看被删除的信息中有没有计算机系的学生的信息
)
begin
    print '不能删除计算机系学生选课信息!'; -- 打印错误信息
    rollback transaction; -- 回滚事务
    return; -- 退出触发器
end
\end{verbatim}
\end{mdframed}

\subsubsection{问题二}

\textbf{题目} \emph{通过创建一个前触发型触发器tr\_stu\_del,限制删除有选课的学生信息。(注意:通过内连接实现,不取别名,临时表在前)}

\vspace{6pt}

\qquad 分析题目可以知道要使用前触发型触发器,在表被修改前进行操作。
根据上一题的解析可以得到答案:

\begin{mdframed}[backgroundcolor=blue!5]
\begin{verbatim}
create trigger tr_stu_del
on stu
instead of delete
as
if exists(
    select 1
    from delete
    inner join sc on delete.sno = sc.sno
    -- 查找要删除的信息中是否存在有选课记录的学生
)
begin
    print '不能删除有选课记录的学生信息!';
    return;
end
else
-- INSTEAD型触发器会拦截删除操作,所以需要手动恢复
begin
    delete from stu
    where sno in (select sno from deleted);
end
\end{verbatim}
\end{mdframed}

\vspace{6pt}

\begin{mdframed}[backgroundcolor=yellow!10]
  \textbf{重要提示:如何判断在哪张表上创建触发器}

  \begin{itemize}
    \item \textbf{看操作对象}:题目中提到要限制对哪张表的操作,触发器就创建在那张表上
    \item \textbf{问题一}:限制删除"计算机系学生的选课信息" → 操作的是SC表 → 在SC表上创建触发器
    \item \textbf{问题二}:限制删除"有选课的学生信息" → 操作的是Student表 → 在Student表上创建触发器
    \item \textbf{一般规律}:
      \begin{itemize}
        \item 如果要控制对表A的INSERT/UPDATE/DELETE操作,就在表A上创建触发器
        \item 触发器会在对该表进行指定操作时自动执行
        \item 通过INSERTED/DELETED表可以获取操作前后的数据进行判断
      \end{itemize}
  \end{itemize}
\end{mdframed}

\subsubsection{问题三}

\textbf{题目} \emph{创建一个用于防止用户删除学生选课数据库中任何数据表的触发器tr\_droptable}

\vspace{6pt}

\qquad 根据题目可以发现,触发器触发条件是删除数据表,由此可以判断需要创建DDL触发器。

\qquad 分析DDL触发器的创建模板:

\begin{mdframed}[backgroundcolor=gray!10]
\begin{verbatim}
create trigger 触发器名
on {all server | database}
[with encryption]
{for | after} {DDL事件} [,...n]
as
    sql_statements
\end{verbatim}
\end{mdframed}

\qquad 每一个DDL事件都对应一个Transact-SQL语句。例如,DROP\_TABLE事件对应DROP TABLE语句,CREATE\_TABLE事件对应CREATE TABLE语句,ALTER\_TABLE事件对应ALTER TABLE语句等。

\qquad 根据题目要求,我们需要防止删除数据表,所以使用DROP\_TABLE事件。答案如下:

\begin{mdframed}[backgroundcolor=blue!5]
\begin{verbatim}
create trigger tr_droptable
on database
for drop_table
as
begin
    print '禁止删除数据库中的表!';
    rollback transaction;
end
\end{verbatim}
\end{mdframed}

\subsubsection{问题四}

\textbf{题目} \emph{为了必须删除一个选课记录(学号95001, 课程号001),请先抑制触发器tr\_sc\_del,删除后,再恢复触发器。}

\vspace{6pt}

\qquad 分析题目可知,要删除一个值需要先禁用触发器,删除后再启用这个触发器,
考察触发器的禁用和启用。

\vspace{6pt}

\textbf{1. 禁用触发器}

\qquad 禁用触发器的模板:

\begin{mdframed}[backgroundcolor=gray!10]
\begin{verbatim}
disable trigger {all | 触发器名 [,...n]} -- 哪个触发器
on {object_name | database | all server} -- 在哪个位置的触发器
\end{verbatim}
\end{mdframed}

\textbf{2. 启用触发器}

\qquad 启用触发器的模板:

\begin{mdframed}[backgroundcolor=gray!10]
\begin{verbatim}
enable trigger {all | 触发器名 [,...n]} -- 哪个触发器
on {object_name | database | all server} -- 在哪个位置的触发器
\end{verbatim}
\end{mdframed}

\qquad 根据题目要求,完整的操作步骤如下:

\begin{mdframed}[backgroundcolor=blue!5]
\begin{verbatim}
-- 第一步:禁用触发器
disable trigger tr_sc_del on sc;

-- 第二步:删除指定的选课记录
delete from sc
where sno = '95001' and cno = '001';

-- 第三步:启用触发器
enable trigger tr_sc_del on sc;
\end{verbatim}
\end{mdframed}

\subsection{知识点拓展}

\subsubsection{查看触发器}

\qquad 可以通过以下几种方式查看触发器:

\begin{enumerate}
  \item \textbf{查看数据库中的所有触发器}:
    \begin{verbatim}
    SELECT * FROM sys.triggers;
    \end{verbatim}

  \item \textbf{查看特定表上的触发器}:
    \begin{verbatim}
    SELECT * FROM sys.triggers
    WHERE parent_id = OBJECT_ID('表名');
    \end{verbatim}

  \item \textbf{查看触发器的详细信息}:
    \begin{verbatim}
    EXEC sp_helptext '触发器名';
    \end{verbatim}
\end{enumerate}

\subsubsection{删除触发器}

\qquad 删除触发器的语法很简单:

\begin{mdframed}[backgroundcolor=gray!10]
\begin{verbatim}
DROP TRIGGER 触发器名;
\end{verbatim}
\end{mdframed}

\qquad 示例:
\begin{verbatim}
-- 删除DML触发器
DROP TRIGGER tr_sc_del;

-- 删除DDL触发器
DROP TRIGGER tr_droptable ON DATABASE;
\end{verbatim}

\textbf{注意:}删除触发器后,该触发器的所有功能将永久失效,请谨慎操作。


\section{第四题:数据表的创建与修改}

\subsection{题目}

\begin{enumerate}
  \item 创建下述course表

    \begin{tabular}{|l|l|l|l|}
      \hline
      列名 & 数据类型 & 宽度 & 为空性 \\
      \hline
      cno & char & 6 & \\
      \hline
      cname & char & 20 & \\
      \hline
      credit & tinyint & & $\surd$ \\
      \hline
    \end{tabular}
  \item 在course表中添加一列semester,整数型,非空
  \item 将cno设置成主键(主键名字为pk\_seme)
  \item 在sc表中将cno设置为外键(外键名字为fk\_cno)
\end{enumerate}

\subsection{解析}

\subsubsection{问题一:创建表}

\textbf{问题} \emph{创建course表}

\qquad 分析创建表的模板:

\begin{mdframed}[backgroundcolor=gray!10]
\begin{verbatim}
CREATE TABLE 表名
(
    列名1 数据类型(长度) [约束条件],
    列名2 数据类型(长度) [约束条件],
    ...
    列名n 数据类型(长度) [约束条件],
    [表级约束]
)
\end{verbatim}
\end{mdframed}

\textbf{常用数据类型:}
\begin{itemize}
  \item \textbf{char(n)}:固定长度字符串
  \item \textbf{varchar(n)}:可变长度字符串
  \item \textbf{int}:整数型
  \item \textbf{tinyint}:小整数型(0-255)
  \item \textbf{decimal(p,s)}:定点数
  \item \textbf{datetime}:日期时间型
\end{itemize}

\textbf{约束条件:}
\begin{itemize}
  \item \textbf{NOT NULL}:非空约束
  \item \textbf{NULL}:允许为空(默认)
  \item \textbf{PRIMARY KEY}:主键约束
  \item \textbf{UNIQUE}:唯一约束
  \item \textbf{DEFAULT 值}:默认值约束
\end{itemize}

\qquad 根据题目要求,创建course表的答案如下:

\begin{mdframed}[backgroundcolor=blue!5]
\begin{verbatim}
CREATE TABLE course
(
    cno CHAR(6) NOT NULL,
    cname CHAR(20) NOT NULL,
    credit TINYINT NULL
)
\end{verbatim}
\end{mdframed}

\subsubsection{问题二:添加列}

\textbf{问题} \emph{在course表中添加一列semester,整数型,非空}

\qquad 分析添加列的模板:

\begin{mdframed}[backgroundcolor=gray!10]
\begin{verbatim}
ALTER TABLE 表名
ADD 列名 数据类型(长度) [约束条件]
\end{verbatim}
\end{mdframed}

\textbf{说明:}
\begin{itemize}
  \item \textbf{ALTER TABLE}:修改表结构的关键字
  \item \textbf{ADD}:添加列的操作
  \item 可以同时添加多列,用逗号分隔
  \item 添加的列默认为NULL,如需非空需显式指定NOT NULL
\end{itemize}

\qquad 根据题目要求,添加semester列的答案如下:

\begin{mdframed}[backgroundcolor=blue!5]
\begin{verbatim}
ALTER TABLE course
ADD semester INT NOT NULL
\end{verbatim}
\end{mdframed}

\subsubsection{问题三:设置主键}

\textbf{问题} \emph{将cno设置成主键(主键名字为pk\_seme)}

\qquad 分析添加主键约束的模板:

\begin{mdframed}[backgroundcolor=gray!10]
\begin{verbatim}
ALTER TABLE 表名
ADD CONSTRAINT 约束名 PRIMARY KEY (列名)
\end{verbatim}
\end{mdframed}

\textbf{说明:}
\begin{itemize}
  \item \textbf{ADD CONSTRAINT}:添加约束的关键字
  \item \textbf{约束名}:自定义的约束名称,便于管理
  \item \textbf{PRIMARY KEY}:主键约束类型
  \item 主键列必须为NOT NULL,且值唯一
  \item 一个表只能有一个主键
\end{itemize}

\qquad 根据题目要求,设置cno为主键的答案如下:

\begin{mdframed}[backgroundcolor=blue!5]
\begin{verbatim}
ALTER TABLE course
ADD CONSTRAINT pk_seme PRIMARY KEY (cno)
\end{verbatim}
\end{mdframed}

\textbf{拓展说明}

\begin{mdframed}[backgroundcolor=yellow!10]
  \textbf{主键的作用:}
  \begin{itemize}
    \item \textbf{唯一性标识}:确保表中每一行都有唯一的标识符,不允许重复值
    \item \textbf{非空约束}:主键列不能包含NULL值,保证数据完整性
    \item \textbf{建立索引}:系统自动为主键创建唯一聚集索引,提高查询性能
    \item \textbf{外键引用}:作为其他表外键的参照目标,建立表间关系
    \item \textbf{数据完整性}:防止插入重复或无效的数据,维护数据质量
    \item \textbf{优化存储}:聚集索引按主键顺序物理存储数据,提高I/O效率
  \end{itemize}
\end{mdframed}

\subsubsection{问题四:设置外键}

\textbf{问题} \emph{在sc表中将cno设置为外键(外键名字为fk\_cno)}

\qquad 外键可以在创建表时定义,也可以在表创建后添加。

\textbf{方式一:创建表时定义外键}

\begin{mdframed}[backgroundcolor=gray!10]
\begin{verbatim}
CREATE TABLE 表名
(
    列名1 数据类型 约束条件,
    列名2 数据类型 约束条件,
    ...
    CONSTRAINT 外键名 FOREIGN KEY (列名)
        REFERENCES 参照表名(参照列名)
)
\end{verbatim}
\end{mdframed}

\textbf{方式二:表创建后添加外键}

\begin{mdframed}[backgroundcolor=gray!10]
\begin{verbatim}
ALTER TABLE 表名
ADD CONSTRAINT 外键名
    FOREIGN KEY (列名) REFERENCES 参照表名(参照列名)
\end{verbatim}
\end{mdframed}

\textbf{说明:}
\begin{itemize}
  \item \textbf{FOREIGN KEY}:定义外键约束
  \item \textbf{REFERENCES}:指定参照的主表和主键
  \item 外键列的数据类型必须与参照列相同
  \item 参照列必须是主键或唯一键
\end{itemize}

\qquad 根据题目要求,在sc表中设置cno为外键的答案如下:

\begin{mdframed}[backgroundcolor=blue!5]
\begin{verbatim}
ALTER TABLE sc
ADD CONSTRAINT fk_cno
    FOREIGN KEY (cno) REFERENCES course(cno)
\end{verbatim}
\end{mdframed}

\begin{mdframed}[backgroundcolor=yellow!10]
  \textbf{外键的作用:}
  \begin{itemize}
    \item \textbf{维护参照完整性}:确保子表中的外键值必须在父表的主键中存在
    \item \textbf{防止数据不一致}:避免插入无效的关联数据
    \item \textbf{级联操作}:可设置级联删除或更新,保持数据同步
    \item \textbf{建立表间关系}:明确表与表之间的逻辑关系
    \item \textbf{约束数据操作}:限制对父表主键的删除和修改操作
    \item \textbf{提高数据可靠性}:防止孤立记录的产生
  \end{itemize}
\end{mdframed}

\subsection{知识点拓展}

\subsubsection{唯一约束}

\qquad 唯一约束确保列中的值是唯一的,但允许NULL值。

\begin{mdframed}[backgroundcolor=gray!10]
\begin{verbatim}
-- 创建表时定义
CREATE TABLE 表名
(
    列名 数据类型 UNIQUE,
    ...
)

-- 后续添加
ALTER TABLE 表名
ADD CONSTRAINT 约束名 UNIQUE (列名)
\end{verbatim}
\end{mdframed}

\textbf{示例:}
\begin{verbatim}
ALTER TABLE student
ADD CONSTRAINT uk_student_email UNIQUE (email);
\end{verbatim}

\subsubsection{检查约束}

\qquad 检查约束用于限制列中允许的值范围。

\begin{mdframed}[backgroundcolor=gray!10]
\begin{verbatim}
-- 创建表时定义
CREATE TABLE 表名
(
    列名 数据类型 CHECK (条件表达式),
    ...
)

-- 后续添加
ALTER TABLE 表名
ADD CONSTRAINT 约束名 CHECK (条件表达式)
\end{verbatim}
\end{mdframed}

\textbf{示例:}
\begin{verbatim}
ALTER TABLE student
ADD CONSTRAINT ck_student_age CHECK (age >= 0 AND age <= 150);
\end{verbatim}

\subsubsection{默认值约束}

\qquad 默认值约束为列提供默认值,在插入数据时如果未指定值则使用默认值。

\begin{mdframed}[backgroundcolor=gray!10]
\begin{verbatim}
-- 创建表时定义
CREATE TABLE 表名
(
    列名 数据类型 DEFAULT 默认值,
    ...
)

-- 后续添加
ALTER TABLE 表名
ADD CONSTRAINT 约束名 DEFAULT 默认值 FOR 列名
\end{verbatim}
\end{mdframed}

\textbf{示例:}
\begin{verbatim}
ALTER TABLE student
ADD CONSTRAINT df_student_status DEFAULT '在读' FOR status;
\end{verbatim}

\subsubsection{约束禁用和启用}

\qquad 可以临时禁用或启用约束,常用于数据导入或维护操作。

\begin{mdframed}[backgroundcolor=gray!10]
\begin{verbatim}
-- 禁用约束
ALTER TABLE 表名 NOCHECK CONSTRAINT 约束名;

-- 启用约束
ALTER TABLE 表名 CHECK CONSTRAINT 约束名;

-- 删除约束
ALTER TABLE 表名 DROP CONSTRAINT 约束名;
\end{verbatim}
\end{mdframed}

\textbf{示例:}
\begin{verbatim}
-- 禁用外键约束
ALTER TABLE sc NOCHECK CONSTRAINT fk_cno;

-- 启用外键约束
ALTER TABLE sc CHECK CONSTRAINT fk_cno;

-- 删除约束
ALTER TABLE sc DROP CONSTRAINT fk_cno;
\end{verbatim}

\begin{mdframed}[backgroundcolor=yellow!10]
  \textbf{约束管理要点:}
  \begin{itemize}
    \item 约束名称应具有描述性,便于识别和管理
    \item 禁用约束后记得及时启用,避免数据完整性问题
    \item 删除约束前要谨慎考虑,确保不会影响数据完整性
    \item 可以通过系统视图查询表中的所有约束信息
  \end{itemize}
\end{mdframed}

\section{第五题:游标的创建与使用}

\subsection{题目}

\qquad 将student表中的同学按照姓名升序后的前7位同学的相关信息打印在消息窗格中,格式为:姓名+‘的年龄为’+年龄。
游标名为stu\_cur,为了方便,将姓名、年龄存储在局部变量为@stu\_name, @stu\_age上。 局部变量声明在打开游标之前。

\subsection{解析}

\qquad 这道题需要创建并打开游标,所以我先介绍一下游标的创建。

\subsubsection{创建游标}

\qquad 创建游标的简单模板

\begin{mdframed}[backgroundcolor=gray!10]
\begin{verbatim}
declare 游标名 cursor
for 数据库查询语句
\end{verbatim}
\end{mdframed}

\qquad \textbf{说明:} \emph{这不是完整的游标创建模板,完整版考试不要求。}

\qquad 简单示例

\begin{mdframed}[backgroundcolor=gray!10]
\begin{verbatim}
declare Mycur cursor
for select Sno,Sname
from student
where Ssex='男'
\end{verbatim}
\end{mdframed}

\subsubsection{打开游标}

\qquad 打开游标的模板

\begin{mdframed}[backgroundcolor=gray!10]
\begin{verbatim}
open [local | global] 游标名 | 游标变量名
\end{verbatim}
\end{mdframed}

\qquad 简单示例

\begin{mdframed}[backgroundcolor=gray!10]
\begin{verbatim}
open Mycur
\end{verbatim}
\end{mdframed}

\subsubsection{读取游标}

\qquad 游标的读取使用FETCH语句,其过程分两步:

\begin{enumerate}
  \item 将游标当前指向的记录保存到一个局部变量中
  \item 游标自动移动到下一条记录
\end{enumerate}

\qquad 当记录读入局部变量后,就可以根据需要进行处理。

\qquad 读取游标的模板

\begin{mdframed}[backgroundcolor=gray!10]
\begin{verbatim}
fetch [[next | prior | first | last |
absolute{n | @nvar | relative{n | @nvar}}]
from ] 游标名 [into @局部变量名 [,...n]]
\end{verbatim}
\end{mdframed}

\qquad 参数含义如\ref{table:cursor:1}。

\begin{table}[h]
  \centering
  \begin{tabular}{|l|l|}
    \hline
    \textbf{参数} & \textbf{含义} \\
    \hline
    NEXT & 移动到下一条记录(默认选项) \\
    \hline
    PRIOR & 移动到上一条记录 \\
    \hline
    FIRST & 移动到第一条记录 \\
    \hline
    LAST & 移动到最后一条记录 \\
    \hline
    ABSOLUTE n & 移动到第n条记录(正数从头开始,负数从尾开始) \\
    \hline
    RELATIVE n & 相对当前位置移动n条记录(正数向前,负数向后) \\
    \hline
    INTO @变量名 & 将获取的数据存储到指定的局部变量中 \\
    \hline
  \end{tabular}
  \caption{FETCH语句参数含义}
  \label{table:cursor:1}
\end{table}

\subsubsection{题目解答}

\qquad 根据题目要求,需要创建游标并获取前7位学生信息。完整代码如下:

\begin{mdframed}[backgroundcolor=blue!5]
\begin{verbatim}
-- 声明局部变量(在打开游标之前)
declare @stu_name varchar(20);
declare @stu_age int;
declare @count int = 0;

-- 声明游标
declare stu_cur cursor
for select sname, sage
    from student
    order by sname asc;

-- 打开游标
open stu_cur;

-- 读取第一条记录
fetch next from stu_cur into @stu_name, @stu_age;

-- 循环处理前7条记录
while @@fetch_status = 0 and @count < 7
begin
    -- 打印信息到消息窗格
    print @stu_name + '的年龄为' + cast(@stu_age as varchar(10));

    -- 计数器加1
    set @count = @count + 1;

    -- 读取下一条记录
    fetch next from stu_cur into @stu_name, @stu_age;
end

-- 关闭游标
close stu_cur;

-- 释放游标
deallocate stu_cur;
\end{verbatim}
\end{mdframed}

\textbf{代码说明:}
\begin{itemize}
  \item \textbf{@@FETCH\_STATUS}:系统变量,表示上一次FETCH操作的状态(0表示成功)
  \item \textbf{CAST函数}:将整数类型的年龄转换为字符串,便于拼接
  \item \textbf{计数器@count}:确保只处理前7条记录
  \item \textbf{CLOSE}:关闭游标,释放资源
  \item \textbf{DEALLOCATE}:释放游标内存
\end{itemize}


\section{第六题:存储过程}

\subsection{题目}

\begin{enumerate}
\item 创建存储过程(p\_stu),实现给定学号(局部变量名为@stu\_sno, 并取默认值为95001),列出年龄大于该同学的学生信息,姓名和年龄(列名不变)。存储过程中的查询语句通过子查询实现。
\item 执行上述存储过程,取95004实验。
\item 创建存储过程(p\_stu2),实现给定学号(局部变量名为@stu\_sno, 并取默认值为95001),列出与该学生属于同一系的其他学生姓名和年龄(原样显示)。存储过程中的查询语句通过自身链接(别名采用s1,s2,且通过s2表返回)实现。
\item 通过95002验证。
\item 创建存储过程p\_count\_cs,根据输入学生学号(@stu\_sno),返回该学生选了多少门课。返回值为整数型,且取名为@c\_ss。
\item 检查学号为95001的情况验证准确性。输出变量取名为@pp\_cnos,打印内容为仅@pp\_cnos。考虑如果是指定课程号,返回有多少学生选了该课程,应该怎么写和验证。
\end{enumerate}

\subsection{解析}

\qquad 在解析题目之前,我先讲一下存储过程的创建和执行。

\subsubsection{存储过程的创建}

\qquad 带参数的存储过程创建简单模板

\begin{mdframed}[backgroundcolor=gray!10]
\begin{verbatim}
create procedure 存储过程名
[{@参数名称 参数数据类型} [ = 参数的默认值]
[output] ] -- 参数后面带output的为输出参数
[,...n]
as
sql_statement
\end{verbatim}
\end{mdframed}

\qquad 简单示例

\begin{mdframed}[backgroundcolor=gray!10]
\begin{verbatim}
-- 带输入参数的存储过程
create procedure p_student
@Sno char(5)
as
select Sname, Sdept
from student where Sno=@Sno

-- 带输出参数的存储过程
create procedure p_sum
@var1 int, @var2 int,
@var3 int output -- 输出参数
as
set @var3 = @var1 * @var2
\end{verbatim}
\end{mdframed}

\qquad 不带参数的存储过程创建简单模板

\begin{mdframed}[backgroundcolor=gray!10]
\begin{verbatim}
create procedure 存储过程名
as
sql_statement
\end{verbatim}
\end{mdframed}

\qquad \textbf{说明:} \emph{完整版模板考试不要求,
\textcolor{Red}{变量前必须加@}。}

\qquad 简单示例

\begin{mdframed}[backgroundcolor=gray!10]
\begin{verbatim}
create procedure p_course
as
select * from course
\end{verbatim}
\end{mdframed}

\subsubsection{存储过程的执行}

\qquad 不带参数的存储过程执行模板

\begin{mdframed}[backgroundcolor=gray!10]
\begin{verbatim}
exec 存储过程名
\end{verbatim}
\end{mdframed}

\qquad 简单示例

\begin{mdframed}[backgroundcolor=gray!10]
\begin{verbatim}
exec p_course
\end{verbatim}
\end{mdframed}

\qquad 带输入参数的存储过程

\begin{mdframed}[backgroundcolor=gray!10]
\begin{verbatim}
exec 存储过程名
[@参数名 = 参数值][default]
[,...n]
\end{verbatim}
\end{mdframed}

\qquad 简单示例

\begin{mdframed}[backgroundcolor=gray!10]
\begin{verbatim}
exec p_grade2 @dept='计算机系'
\end{verbatim}
\end{mdframed}

\qquad 带输出参数的存储过程

\begin{mdframed}[backgroundcolor=gray!10]
\begin{verbatim}
exec 存储过程名
[[@参数名=]{参数值 | @变量[output] | [默认值]}][,...n]
\end{verbatim}
\end{mdframed}

\qquad 简单示例

\begin{mdframed}[backgroundcolor=gray!10]
\begin{verbatim}
declare @res int
exec p_sum @var1=3,@var2=8,@res=@res output

-- 或者使用参数顺序调用
exec p_sum 3,8,@res output
\end{verbatim}
\end{mdframed}

\subsubsection{问题一}

\textbf{题目} \emph{创建存储过程(p\_stu),实现给定学号(局部变量名为@stu\_sno, 并取默认值为95001),列出年龄大于该同学的学生信息,姓名和年龄(列名不变)。存储过程中的查询语句通过子查询实现。}

\qquad 根据存储过程的知识可得到答案。

\begin{mdframed}[backgroundcolor=blue!5]
\begin{verbatim}
create procedure p_stu
@stu_sno char(5) = '95001'
as
select Sname, Sage
from student
where Sage > (
    select Sage from student
    where Sno=@stu_sno
)
\end{verbatim}
\end{mdframed}

\subsubsection{问题二}

\textbf{题目} \emph{执行上述存储过程,取95004实验。}

\vspace{6pt}

\qquad 传入95004作为参数执问题一的存储过程,得到答案。

\begin{mdframed}[backgroundcolor=blue!5]
\begin{verbatim}
exec p_stu '95004' -- 注意char类型需要加引号
\end{verbatim}
\end{mdframed}

\subsubsection{问题三}

\textbf{题目} \emph{创建存储过程(p\_stu2),实现给定学号(局部变量名为@stu\_sno, 并取默认值为95001),列出与该学生属于同一系的其他学生姓名和年龄(原样显示)。存储过程中的查询语句通过自身链接(别名采用s1,s2,且通过s2表返回)实现。}

\vspace{6pt}

\qquad 难点在于如何查询。得到答案。

\begin{mdframed}[backgroundcolor=blue!5]
\begin{verbatim}
create procedure p_stu2
@stu_sno char(5) = '95001'
as
select s2.Sname, s2.Sage
from student s2 join student s1
on s2.Sdept=s1.Sdept
where s1.Sno=@stu_sno and s2.Sno != @stu_sno
\end{verbatim}
\end{mdframed}

\qquad \textbf{注意:} \textcolor{red}{本题要查询的是“其他学生”,不包括输入的学生,注意排除。}

\subsubsection{问题四}

\textbf{题目} \emph{通过95002验证。}

\vspace{6pt}

\qquad 执行时传入参数即可得到答案。

\begin{mdframed}[backgroundcolor=blue!5]
\begin{verbatim}
exec p_stu2 '95002'
\end{verbatim}
\end{mdframed}

\subsubsection{问题五}

\textbf{题目} \emph{创建存储过程p\_count\_cs,根据输入学生学号(@stu\_sno),返回该学生选了多少门课。返回值为整数型,且取名为@c\_ss。}

\vspace{6pt}

\qquad 这道题的考点在于存储过程的输出参数。得到答案。

\begin{mdframed}[backgroundcolor=blue!5]
\begin{verbatim}
create procedure p_count_cs
@stu_sno char(5), @c_ss int output
as
select @c_ss=count(*)
from SC
where Sno=@stu_sno
\end{verbatim}
\end{mdframed}

\subsubsection{问题六}

\textbf{题目} \emph{检查学号为95001的情况验证准确性。输出变量取名为@pp\_cnos,打印内容为仅@pp\_cnos。考虑如果是指定课程号,返回有多少学生选了该课程,应该怎么写和验证。}

\vspace{6pt}

\qquad 这题的考点是带参数的存储过程的执行。得到答案。

\begin{mdframed}[backgroundcolor=blue!5]
\begin{verbatim}
-- 定义变量
declare @pp_cnos int

-- 执行存储过程
exec p_count_cs '95001', @pp_cnos output

-- 打印结果
print @pp_cnos
\end{verbatim}
\end{mdframed}

\textbf{注意:} \textcolor{red}{输出参数的后面要加output,不要忘记。}

\subsection{知识点拓展}

\subsubsection{查看存储过程}

\qquad 可以通过以下几种方式查看存储过程:

\begin{enumerate}
\item \textbf{查看所有存储过程}:
    \begin{verbatim}
    SELECT * FROM sys.procedures;
    \end{verbatim}

\item \textbf{查看存储过程定义}:
    \begin{verbatim}
    EXEC sp_helptext '存储过程名';
    \end{verbatim}

\item \textbf{查看存储过程参数}:
    \begin{verbatim}
    EXEC sp_help '存储过程名';
    \end{verbatim}
\end{enumerate}

\textbf{示例:}
\begin{verbatim}
EXEC sp_helptext 'p_stu';
\end{verbatim}

\subsubsection{删除用户存储过程}

\qquad 删除存储过程的语法:

\begin{mdframed}[backgroundcolor=gray!10]
\begin{verbatim}
DROP PROCEDURE 存储过程名;
\end{verbatim}
\end{mdframed}

\textbf{示例:}
\begin{verbatim}
-- 删除单个存储过程
DROP PROCEDURE p_stu;

-- 删除多个存储过程
DROP PROCEDURE p_stu, p_stu2, p_count_cs;
\end{verbatim}

\textbf{注意:}删除存储过程是不可逆操作,请谨慎执行。

\subsubsection{修改存储过程}

\qquad 修改存储过程的语法:

\begin{mdframed}[backgroundcolor=gray!10]
\begin{verbatim}
ALTER PROCEDURE 存储过程名
[参数列表]
AS
BEGIN
    -- 修改后的SQL语句
END
\end{verbatim}
\end{mdframed}

\textbf{示例:}
\begin{lstlisting}[backgroundcolor=\color{Gray!10}]
ALTER PROCEDURE p_stu
@stu_sno CHAR(5) = '95001'
AS
BEGIN
    SELECT Sname, Sage, Sdept  -- 增加系别信息
    FROM student
    WHERE Sage > (
        SELECT Sage FROM student
        WHERE Sno = @stu_sno
    )
END
\end{lstlisting}

\begin{mdframed}[backgroundcolor=yellow!10]
\textbf{存储过程管理要点:}
\begin{itemize}
\item 修改存储过程时使用ALTER PROCEDURE,而不是重新CREATE
\item 存储过程名在数据库中必须唯一
\item 定期检查和优化存储过程的性能
\item 为存储过程添加适当的注释,便于维护
\item 测试存储过程的各种参数组合,确保逻辑正确
\end{itemize}
\end{mdframed}

\section{第七题:安全管理}

\subsection{题目}

\begin{enumerate}
  \item 请建立SQL Server登录名sql\_user1,并映射至同名用户名。其中登录密码为'nulibeikao'。
  \item 对象权限,授予sql\_user1用户在student表中的插入,更新,查询权利。
  \item 语句权限,允许sql\_user1用户在数据库上创建视图、存储过程的权限。
  \item 语句权限,拒绝sql\_user1用户在数据库上创建表的权限
  \item 对象权限,拒绝sql\_user1用户在sc表上删除数据的权利
\end{enumerate}

\subsection{解析}

\qquad 这一章的内容基本都可以通过SSMS软件进行图形化操作,详见书本第九章P202。
这里主要介绍如何通过指令操作。

\subsubsection{问题一}

\textbf{题目} \emph{请建立SQL Server登录名sql\_user1,并映射至同名用户名。其中登录密码为'nulibeikao'。}

\qquad 分析创建和映射数据库用户的简单模板。

\begin{mdframed}[backgroundcolor=gray!10]
\begin{verbatim}
-- 创建数据库用户
create user 用户名
[with Password=密码]
[DEFAULT_DATABASE=默认数据库]

-- 创建数据库登录名
create login 登录名
[with Password=密码]

-- 映射登录名到数据库用户
create user 用户名
for login 登录名
\end{verbatim}
\end{mdframed}

\qquad \textbf{用户名、登录名、映射关系简介:}

\begin{itemize}
  \item \textbf{登录名(Login)}:是SQL Server服务器级别的安全主体,用于连接到SQL Server实例。登录名存储在master数据库中,是进入SQL Server的"钥匙"。

  \item \textbf{用户名(User)}:是数据库级别的安全主体,存在于特定的数据库中。用户名决定了在该数据库内能执行哪些操作。

  \item \textbf{映射关系}:登录名和用户名之间需要建立映射关系,这样登录名才能访问特定的数据库。一个登录名可以映射到多个数据库中的不同用户名。一个登录名在一个数据库中只能有有唯一一个数据库用户与之对应。

  \item \textbf{权限层次}:
    \begin{itemize}
      \item 服务器级权限 $\rightarrow$ 登录名
      \item 数据库级权限 $\rightarrow$ 用户名
    \end{itemize}
\end{itemize}

\qquad 简单来说:\textbf{登录名负责"进门",用户名负责"干活"}。先用登录名连接到SQL Server,再通过映射的用户名在具体数据库中执行操作。

\qquad 根据模板可以得到答案。

\begin{mdframed}[backgroundcolor=blue!5]
\begin{verbatim}
-- 创建登录名
create login sql_user1
with Password='nulibeikao'

-- 映射登录名到用户
create user sql_user1
for login sql_user1
\end{verbatim}
\end{mdframed}

\textbf{注意:} \textcolor{Red}{如果题目中要求建立映射关系,但前面没有建立登录名,要先建立登录名再建立映射关系。}

\subsubsection{问题二}

\textbf{题目} \emph{对象权限,授予sql\_user1用户在student表中的插入,更新,查询权利。}

\vspace{6pt}

\qquad 这道题涉及到数据库用户权限管理,我先介绍一下数据库的权限类别,如表\ref{table:2}。

\begin{table}[H]
  \centering
  \begin{tabularx}{\textwidth}{p{5cm}|X}
    \toprule
    \textbf{权限类别} & \textbf{描述} \\
    \hline
    SELECT & 查询权限,允许用户查看表或视图中的数据 \\
    INSERT & 插入权限,允许用户向表中添加新的数据行 \\
    UPDATE & 更新权限,允许用户修改表中现有的数据 \\
    DELETE & 删除权限,允许用户删除表中的数据行 \\
    REFERENCES & 引用权限,允许用户创建引用该表的外键约束 \\
    ALTER & 修改权限,允许用户修改表结构(添加/删除列等) \\
    INDEX & 索引权限,允许用户在表上创建或删除索引 \\
    EXECUTE & 执行权限,允许用户执行存储过程或函数 \\
    CREATE TABLE & 创建表权限,允许用户在数据库中创建新表 \\
    CREATE VIEW & 创建视图权限,允许用户在数据库中创建视图 \\
    CREATE PROCEDURE & 创建存储过程权限,允许用户创建存储过程 \\
    \bottomrule
  \end{tabularx}
  \caption{SQL Server权限类别说明}
  \label{table:2}
\end{table}

\qquad 下面给出授予用户权限的模板并给出几个例子。

\begin{mdframed}[backgroundcolor=gray!10]
\begin{verbatim}
-- 模板
grant 权限类别[,...n] [on 表名[,...n]] to 用户名

-- 授予用户创建数据库的权限
grant create database to Qtcyy

-- 授予用户对student表进行插入、更新数据的权限
grant insert, update on student to Qtcyy
\end{verbatim}
\end{mdframed}

\qquad 根据模板可以得到答案。

\begin{mdframed}[backgroundcolor=blue!5]
\begin{verbatim}
grant insert, update, select on student to sql_user1
\end{verbatim}
\end{mdframed}

\textbf{老师提醒:} \textcolor{Red}{注意题目的各个权限的顺序,考试系统内标准比较严,请按照顺序给。}

\subsubsection{问题三}

\textbf{题目} \emph{语句权限,允许sql\_user1用户在数据库上创建视图、存储过程的权限。}

\vspace{6pt}

\qquad 这题考查的是创建视图、存储过程的权限,根据模板可以得到答案。

\begin{mdframed}[backgroundcolor=blue!5]
\begin{verbatim}
grant create view, create procedure to sql_user1
\end{verbatim}
\end{mdframed}

\subsubsection{问题四}

\textbf{题目} \emph{语句权限,拒绝sql\_user1用户在数据库上创建表的权限}

\vspace{6pt}

\qquad 这道题涉及到拒绝用户权限,需要使用DENY语句。下面给出拒绝用户权限的模板。

\begin{mdframed}[backgroundcolor=gray!10]
\begin{verbatim}
-- 拒绝权限模板
deny 权限类别[,...n] [on 表名[,...n]] to 用户名

-- 拒绝用户创建表的权限
deny create table to Qtcyy

-- 拒绝用户对student表进行删除数据的权限
deny delete on student to Qtcyy
\end{verbatim}
\end{mdframed}

\qquad 根据模板可以得到答案。

\begin{mdframed}[backgroundcolor=blue!5]
\begin{verbatim}
deny create table to sql_user1
\end{verbatim}
\end{mdframed}

\qquad \textbf{权限控制语句对比:}

\begin{itemize}
  \item \textbf{GRANT}:授予权限,允许用户执行某些操作
  \item \textbf{DENY}:拒绝权限,明确禁止用户执行某些操作
  \item \textbf{REVOKE}:撤销权限,取消之前授予或拒绝的权限
\end{itemize}

\textbf{注意:} \textcolor{Red}{DENY的优先级高于GRANT,即使用户通过其他角色获得了权限,DENY仍然有效。}

\subsubsection{问题五}

\textbf{题目} \emph{对象权限,拒绝sql\_user1用户在sc表上删除数据的权利}

\vspace{6pt}

\qquad 根据模板可以得到答案。

\begin{mdframed}[backgroundcolor=blue!5]
\begin{verbatim}
deny delete on sc to sql_user1
\end{verbatim}
\end{mdframed}

\subsection{知识点拓展}

\subsubsection{身份验证模式}

\qquad SQL Server支持两种身份验证模式:

\begin{itemize}
  \item \textbf{Windows身份验证模式}:
    \begin{itemize}
      \item 使用Windows用户账户或Windows组账户进行身份验证
      \item 更安全,因为利用了Windows的安全机制
      \item 支持密码策略、账户锁定等Windows安全功能
      \item 适用于企业内部网络环境
    \end{itemize}

  \item \textbf{混合模式(SQL Server和Windows身份验证)}:
    \begin{itemize}
      \item 同时支持Windows身份验证和SQL Server身份验证
      \item SQL Server身份验证使用存储在SQL Server中的登录名和密码
      \item 适用于需要支持非Windows客户端的环境
      \item 默认启用sa账户(系统管理员账户)
    \end{itemize}
\end{itemize}

\subsubsection{服务器角色}

\qquad SQL Server提供了预定义的服务器级固定角色,用于管理服务器级权限:

\begin{table}[H]
  \centering
  \begin{tabularx}{\textwidth}{p{4cm}|X}
    \toprule
    \textbf{服务器角色} & \textbf{权限描述} \\
    \hline
    sysadmin & 系统管理员,拥有服务器的完全控制权限 \\
    serveradmin & 服务器管理员,可以更改服务器范围的配置选项和关闭服务器 \\
    securityadmin & 安全管理员,管理登录名和权限,可以重置密码 \\
    processadmin & 进程管理员,可以终止在SQL Server实例中运行的进程 \\
    setupadmin & 安装管理员,可以添加和删除链接服务器 \\
    bulkadmin & 批量管理员,可以执行BULK INSERT语句 \\
    diskadmin & 磁盘管理员,管理磁盘文件 \\
    dbcreator & 数据库创建者,可以创建、修改、删除和还原任何数据库 \\
    public & 公共角色,每个SQL Server登录名都属于此角色 \\
    \bottomrule
  \end{tabularx}
  \caption{SQL Server服务器角色}
  \label{table:3}
\end{table}

\qquad \textbf{添加用户到服务器角色的语法:}

\begin{mdframed}[backgroundcolor=gray!10]
\begin{verbatim}
-- 将登录名添加到服务器角色
ALTER SERVER ROLE 角色名 ADD MEMBER 登录名

-- 示例:将sql_user1添加到dbcreator角色
ALTER SERVER ROLE dbcreator ADD MEMBER sql_user1
\end{verbatim}
\end{mdframed}

\subsubsection{数据库角色}

\qquad 数据库角色用于管理数据库级别的权限,包括固定数据库角色和用户定义角色:

\begin{table}[H]
  \centering
  \begin{tabularx}{\textwidth}{p{4cm}|X}
    \toprule
    \textbf{数据库角色} & \textbf{权限描述} \\
    \hline
    db\_owner & 数据库所有者,拥有数据库的完全控制权限 \\
    db\_accessadmin & 访问管理员,可以添加或删除数据库用户 \\
    db\_securityadmin & 安全管理员,可以管理角色成员身份和权限 \\
    db\_ddladmin & DDL管理员,可以运行任何DDL命令(CREATE、ALTER、DROP) \\
    db\_backupoperator & 备份操作员,可以备份数据库 \\
    db\_datareader & 数据读取者,可以从所有用户表中读取数据 \\
    db\_datawriter & 数据写入者,可以在所有用户表中添加、更新或删除数据 \\
    db\_denydatareader & 拒绝数据读取者,不能读取数据库中的任何数据 \\
    db\_denydatawriter & 拒绝数据写入者,不能修改数据库中的任何数据 \\
    public & 公共角色,每个数据库用户都属于此角色 \\
    \bottomrule
  \end{tabularx}
  \caption{SQL Server数据库角色}
  \label{table:4}
\end{table}

\qquad \textbf{数据库角色管理语法:}

\begin{mdframed}[backgroundcolor=gray!10]
\begin{verbatim}
-- 将用户添加到数据库角色
ALTER ROLE 角色名 ADD MEMBER 用户名

-- 从数据库角色中删除用户
ALTER ROLE 角色名 DROP MEMBER 用户名

-- 创建用户定义的数据库角色
CREATE ROLE 角色名

-- 示例:将sql_user1添加到db_datareader角色
ALTER ROLE db_datareader ADD MEMBER sql_user1
\end{verbatim}
\end{mdframed}

\qquad \textbf{权限继承关系:}

\begin{itemize}
  \item 用户可以同时属于多个角色
  \item 用户获得所属所有角色的权限(权限的并集)
  \item DENY权限始终优先于GRANT权限
  \item 角色权限与直接授予用户的权限相互补充
\end{itemize}

\section{第八题:数据库的备份与恢复}

\subsection{题目}

\begin{enumerate}
  \item 创建销售管理数据库的完整备份,备份设备名为abc+学号后3位,如:abc109(备份设备处截图),位置放置在D:\textbackslash abc。
  \item 插入一个新员工(自己的名字),然后进行一次差异备份。
  \item 删除刚添加的新员工信息,再进行一次日志备份。
  \item 恢复完整备份。
  \item 恢复完全备份+差异备份。(或者换一种问法,即,恢复到刚刚插入新员工的位置)
  \item 恢复完全备份+差异备份+日志备份。(或者换一种问法,即,恢复到刚刚删除新员工的位置)
\end{enumerate}

\subsection{解析}
\qquad 这个题涉及数据库的备份和恢复备份,我会介绍一下数据库备份与恢复相关的知识点。

\subsubsection{创建备份设备}

\qquad 备份设备是SQL Server中用于存储备份文件的逻辑设备,可以是磁盘文件或磁带设备。通过创建备份设备,可以简化备份操作,避免每次都指定完整的文件路径。

\qquad \textbf{模板:}
\begin{mdframed}[backgroundcolor=gray!10]
\begin{verbatim}
-- 创建备份设备模板
EXEC sp_addumpdevice
    @devtype = '[设备类型:disk/tape]',
    @logicalname = '[备份设备逻辑名称]',
    @physicalname = '[备份文件完整路径]'
\end{verbatim}
\end{mdframed}

\qquad \textbf{示例:}
\begin{mdframed}[backgroundcolor=blue!5]
\begin{verbatim}
-- 为销售管理数据库创建备份设备abc109
EXEC sp_addumpdevice
    @devtype = 'disk',
    @logicalname = 'abc109',
    @physicalname = 'D:\abc\abc109.bak'
\end{verbatim}
\end{mdframed}

\subsubsection{备份类型}

\textbf{1. 完整备份}

\qquad \textbf{概念说明:}完整备份会备份整个数据库,包括所有数据页、索引页、系统表和数据库结构信息。它是所有其他备份类型的基础。

\qquad \textbf{特点:}
\begin{itemize}
  \item \textbf{优点}:包含完整的数据库信息,恢复时不依赖其他备份文件
  \item \textbf{缺点}:备份时间长,占用存储空间大
  \item \textbf{适用场景}:定期的基础备份,作为其他备份类型的基础
  \item \textbf{恢复能力}:可以独立恢复到备份时的状态
\end{itemize}

\qquad \textbf{模板:}
\begin{mdframed}[backgroundcolor=gray!10]
\begin{verbatim}
-- 完整备份模板
BACKUP DATABASE [数据库名称]
TO [备份设备名称或DISK='文件路径']
WITH FORMAT, INIT,
NAME = '[备份名称描述]',
SKIP, NOREWIND, NOUNLOAD, STATS = 10
\end{verbatim}
\end{mdframed}

\qquad \textbf{示例:}
\begin{mdframed}[backgroundcolor=blue!5]
\begin{verbatim}
-- 销售管理数据库完整备份
BACKUP DATABASE 销售管理 TO abc109
WITH FORMAT, INIT,
NAME = '销售管理-完整数据库备份',
SKIP, NOREWIND, NOUNLOAD, STATS = 10
\end{verbatim}
\end{mdframed}

\textbf{2. 差异备份}

\qquad \textbf{概念说明:}差异备份只备份自上次完整备份以来发生变化的数据页。它基于数据库的差异位图(Differential Bitmap)来识别哪些页面发生了变化。

\qquad \textbf{特点:}
\begin{itemize}
  \item \textbf{优点}:备份速度快,占用空间小,备份时间可预测
  \item \textbf{缺点}:恢复时需要完整备份和最新的差异备份,随时间推移备份会越来越大
  \item \textbf{适用场景}:数据变化频繁但希望快速备份的环境
  \item \textbf{恢复能力}:需要与完整备份配合使用
\end{itemize}

\qquad \textbf{模板:}
\begin{mdframed}[backgroundcolor=gray!10]
\begin{verbatim}
-- 差异备份模板
BACKUP DATABASE [数据库名称]
TO [备份设备名称或DISK='文件路径']
WITH DIFFERENTIAL,
NAME = '[备份名称描述]',
SKIP, NOREWIND, NOUNLOAD, STATS = 10
\end{verbatim}
\end{mdframed}

\qquad \textbf{示例:}
\begin{mdframed}[backgroundcolor=blue!5]
\begin{verbatim}
-- 插入新员工后进行差异备份
INSERT INTO 员工表 VALUES ('张三', '开发部', 8000)

BACKUP DATABASE 销售管理
TO DISK = 'D:\abc\销售管理_差异备份.bak'
WITH DIFFERENTIAL,
NAME = '销售管理-差异数据库备份',
SKIP, NOREWIND, NOUNLOAD, STATS = 10
\end{verbatim}
\end{mdframed}

\textbf{3. 事务日志备份}

\qquad \textbf{概念说明:}事务日志备份用于备份事务日志文件中的活动事务记录,记录自上次日志备份以来的所有数据库活动。它是实现时点恢复的关键。

\qquad \textbf{特点:}
\begin{itemize}
  \item \textbf{优点}:备份速度最快,文件最小,支持时点恢复,数据丢失最少
  \item \textbf{缺点}:需要数据库处于完整恢复模式,必须定期执行以防日志文件过大
  \item \textbf{适用场景}:对数据安全要求极高,需要频繁备份的环境
  \item \textbf{恢复能力}:可以恢复到任意时间点,最大限度减少数据丢失
\end{itemize}

\qquad \textbf{模板:}
\begin{mdframed}[backgroundcolor=gray!10]
\begin{verbatim}
-- 事务日志备份模板(需要完整恢复模式)
ALTER DATABASE [数据库名称] SET RECOVERY FULL

BACKUP LOG [数据库名称]
TO [备份设备名称或DISK='文件路径']
WITH NAME = '[备份名称描述]',
SKIP, NOREWIND, NOUNLOAD, STATS = 10
\end{verbatim}
\end{mdframed}

\qquad \textbf{示例:}
\begin{mdframed}[backgroundcolor=blue!5]
\begin{verbatim}
-- 删除员工后进行日志备份
DELETE FROM 员工表 WHERE 姓名 = '张三'

-- 设置完整恢复模式
ALTER DATABASE 销售管理 SET RECOVERY FULL

-- 进行日志备份
BACKUP LOG 销售管理
TO DISK = 'D:\abc\销售管理_日志备份.trn'
WITH NAME = '销售管理-事务日志备份',
SKIP, NOREWIND, NOUNLOAD, STATS = 10
\end{verbatim}
\end{mdframed}

\subsubsection{数据库恢复}

\textbf{1. 恢复完整备份}

\qquad \textbf{说明:}这是最简单的恢复方式,直接从完整备份恢复整个数据库。会丢失备份时间点之后的所有数据变更。

\qquad \textbf{模板:}
\begin{mdframed}[backgroundcolor=gray!10]
\begin{verbatim}
-- 恢复完整备份模板
RESTORE DATABASE [数据库名称]
FROM [备份设备名称或DISK='文件路径']
WITH REPLACE, RECOVERY
\end{verbatim}
\end{mdframed}

\qquad \textbf{示例:}
\begin{mdframed}[backgroundcolor=blue!5]
\begin{verbatim}
-- 方式1:从备份设备恢复
RESTORE DATABASE 销售管理 FROM abc109
WITH REPLACE, RECOVERY

-- 方式2:从文件路径恢复
RESTORE DATABASE 销售管理
FROM DISK = 'D:\abc\abc109.bak'
WITH REPLACE, RECOVERY
\end{verbatim}
\end{mdframed}

\textbf{2. 恢复完整备份+差异备份}

\qquad \textbf{说明:}先恢复完整备份,再应用差异备份。可以恢复到差异备份时的状态,减少了部分数据丢失。注意必须使用NORECOVERY选项,直到最后一步才使用RECOVERY。

\qquad \textbf{模板:}
\begin{mdframed}[backgroundcolor=gray!10]
\begin{verbatim}
-- 恢复完整备份+差异备份模板
RESTORE DATABASE [数据库名称]
FROM [完整备份设备或路径]
WITH REPLACE, NORECOVERY

RESTORE DATABASE [数据库名称]
FROM [差异备份设备或路径]
WITH RECOVERY
\end{verbatim}
\end{mdframed}

\qquad \textbf{示例:}
\begin{mdframed}[backgroundcolor=blue!5]
\begin{verbatim}
-- 恢复到插入新员工后的状态
-- 方式1:从备份设备恢复完整备份
RESTORE DATABASE 销售管理 FROM abc109
WITH REPLACE, NORECOVERY

RESTORE DATABASE 销售管理
FROM DISK = 'D:\abc\销售管理_差异备份.bak'
WITH RECOVERY

-- 方式2:完全使用文件路径
RESTORE DATABASE 销售管理
FROM DISK = 'D:\abc\abc109.bak'
WITH REPLACE, NORECOVERY

RESTORE DATABASE 销售管理
FROM DISK = 'D:\abc\销售管理_差异备份.bak'
WITH RECOVERY
\end{verbatim}
\end{mdframed}

\textbf{3. 恢复完整备份+差异备份+日志备份}

\qquad \textbf{说明:}这是最完整的恢复策略,可以恢复到日志备份时的精确状态。通过应用事务日志,可以实现几乎零数据丢失的恢复。

\qquad \textbf{模板:}
\begin{mdframed}[backgroundcolor=gray!10]
\begin{verbatim}
-- 恢复完整+差异+日志备份模板
RESTORE DATABASE [数据库名称]
FROM [完整备份设备或路径]
WITH REPLACE, NORECOVERY

RESTORE DATABASE [数据库名称]
FROM [差异备份设备或路径]
WITH NORECOVERY

RESTORE LOG [数据库名称]
FROM [日志备份设备或路径]
WITH RECOVERY
\end{verbatim}
\end{mdframed}

\qquad \textbf{示例:}
\begin{mdframed}[backgroundcolor=blue!5]
\begin{verbatim}
-- 恢复到删除员工后的状态
-- 方式1:混合使用备份设备和文件路径
RESTORE DATABASE 销售管理 FROM abc109
WITH REPLACE, NORECOVERY

RESTORE DATABASE 销售管理
FROM DISK = 'D:\abc\销售管理_差异备份.bak'
WITH NORECOVERY

RESTORE LOG 销售管理
FROM DISK = 'D:\abc\销售管理_日志备份.trn'
WITH RECOVERY

-- 方式2:完全使用文件路径
RESTORE DATABASE 销售管理
FROM DISK = 'D:\abc\abc109.bak'
WITH REPLACE, NORECOVERY

RESTORE DATABASE 销售管理
FROM DISK = 'D:\abc\销售管理_差异备份.bak'
WITH NORECOVERY

RESTORE LOG 销售管理
FROM DISK = 'D:\abc\销售管理_日志备份.trn'
WITH RECOVERY
\end{verbatim}
\end{mdframed}

\subsubsection{问题一}

\textbf{题目} \emph{创建销售管理数据库的完整备份,备份设备名为abc+学号后3位,如:abc109(备份设备处截图),位置放置在D:\textbackslash abc。}

\qquad 这个问题需要先创建备份设备,然后进行完整备份。备份设备的作用是为备份文件创建一个逻辑名称,简化后续的备份和恢复操作。

\qquad 根据前面的创建备份设备和完整备份模板可得到答案:

\begin{mdframed}[backgroundcolor=blue!5]
\begin{verbatim}
-- 步骤1:创建备份设备abc109
EXEC sp_addumpdevice
    @devtype = 'disk',
    @logicalname = 'abc109',
    @physicalname = 'D:\abc\abc109.bak'

-- 步骤2:执行销售管理数据库的完整备份
BACKUP DATABASE 销售管理 TO abc109
WITH FORMAT, INIT,
NAME = '销售管理数据库完整备份',
SKIP, NOREWIND, NOUNLOAD, STATS = 10
\end{verbatim}
\end{mdframed}

\subsubsection{问题二}

\textbf{题目} \emph{插入一个新员工(自己的名字),然后进行一次差异备份。}

\qquad 差异备份只备份自上次完整备份以来发生变化的数据页。在插入新员工后进行差异备份,可以快速保存这个新增的数据变更。

\qquad 根据前面的差异备份模板可得到答案:

\begin{mdframed}[backgroundcolor=blue!5]
\begin{verbatim}
-- 步骤1:插入新员工(假设员工表包含姓名、部门、工资字段)
INSERT INTO 员工表 (姓名, 部门, 工资)
VALUES ('张三', '开发部', 8000)

-- 步骤2:执行差异备份
BACKUP DATABASE 销售管理
TO DISK = 'D:\abc\销售管理_差异备份.bak'
WITH DIFFERENTIAL,
NAME = '销售管理数据库差异备份-插入新员工后',
SKIP, NOREWIND, NOUNLOAD, STATS = 10
\end{verbatim}
\end{mdframed}

\subsubsection{问题三}

\textbf{题目} \emph{删除刚添加的新员工信息,再进行一次日志备份。}

\qquad 日志备份用于备份事务日志中的活动记录,记录删除操作等数据库变更。执行日志备份前需要确保数据库处于完整恢复模式。

\qquad 根据前面的事务日志备份模板可得到答案:

\begin{mdframed}[backgroundcolor=blue!5]
\begin{verbatim}
-- 步骤1:删除刚添加的员工
DELETE FROM 员工表 WHERE 姓名 = '张三'

-- 步骤2:设置数据库为完整恢复模式(如果尚未设置)
ALTER DATABASE 销售管理 SET RECOVERY FULL

-- 步骤3:执行日志备份
BACKUP LOG 销售管理
TO DISK = 'D:\abc\销售管理_日志备份.trn'
WITH NAME = '销售管理数据库日志备份-删除员工后',
SKIP, NOREWIND, NOUNLOAD, STATS = 10
\end{verbatim}
\end{mdframed}

\subsubsection{问题四}

\textbf{题目} \emph{恢复完整备份。}

\qquad 这是最基本的恢复操作,直接从完整备份恢复整个数据库到备份时的状态。

\qquad 根据前面的恢复完整备份模板可得到答案:

\begin{mdframed}[backgroundcolor=blue!5]
\begin{verbatim}
-- 方式1:从备份设备恢复
RESTORE DATABASE 销售管理 FROM abc109
WITH REPLACE, RECOVERY

-- 方式2:从文件路径恢复
RESTORE DATABASE 销售管理
FROM DISK = 'D:\abc\abc109.bak'
WITH REPLACE, RECOVERY
\end{verbatim}
\end{mdframed}

\subsubsection{问题五}

\textbf{题目} \emph{恢复完全备份+差异备份。(或者换一种问法,即,恢复到刚刚插入新员工的位置)}

\qquad 这种恢复方式可以恢复到差异备份时的状态,即插入新员工之后、删除员工之前的状态。

\qquad 根据前面的恢复完整备份+差异备份模板可得到答案:

\begin{mdframed}[backgroundcolor=blue!5]
\begin{verbatim}
-- 方式1:使用备份设备恢复完整备份
RESTORE DATABASE 销售管理 FROM abc109
WITH REPLACE, NORECOVERY

RESTORE DATABASE 销售管理
FROM DISK = 'D:\abc\销售管理_差异备份.bak'
WITH RECOVERY

-- 方式2:完全使用文件路径
RESTORE DATABASE 销售管理
FROM DISK = 'D:\abc\abc109.bak'
WITH REPLACE, NORECOVERY

RESTORE DATABASE 销售管理
FROM DISK = 'D:\abc\销售管理_差异备份.bak'
WITH RECOVERY
\end{verbatim}
\end{mdframed}

\subsubsection{问题六}

\textbf{题目} \emph{恢复完全备份+差异备份+日志备份。(或者换一种问法,即,恢复到刚刚删除新员工的位置)}

\qquad 这是最完整的恢复策略,可以恢复到日志备份时的精确状态,即删除员工操作完成后的状态。

\qquad 根据前面的恢复完整备份+差异备份+日志备份模板可得到答案:

\begin{mdframed}[backgroundcolor=blue!5]
\begin{verbatim}
-- 方式1:混合使用备份设备和文件路径
RESTORE DATABASE 销售管理 FROM abc109
WITH REPLACE, NORECOVERY

RESTORE DATABASE 销售管理
FROM DISK = 'D:\abc\销售管理_差异备份.bak'
WITH NORECOVERY

RESTORE LOG 销售管理
FROM DISK = 'D:\abc\销售管理_日志备份.trn'
WITH RECOVERY

-- 方式2:完全使用文件路径
RESTORE DATABASE 销售管理
FROM DISK = 'D:\abc\abc109.bak'
WITH REPLACE, NORECOVERY

RESTORE DATABASE 销售管理
FROM DISK = 'D:\abc\销售管理_差异备份.bak'
WITH NORECOVERY

RESTORE LOG 销售管理
FROM DISK = 'D:\abc\销售管理_日志备份.trn'
WITH RECOVERY
\end{verbatim}
\end{mdframed}

\end{document}